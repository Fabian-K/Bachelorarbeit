%--------------------------------------------------------------------------------------
% Trennungsregeln
%--------------------------------------------------------------------------------------
\hyphenation{prob-lem-los}

%--------------------------------------------------------------------------------------
% Abkürzungen
%--------------------------------------------------------------------------------------
\newacronym{HANA}{{HANA}}{SAP High Performance Analytic Appliance}
\newacronym{DBMS}{DBMS}{Datenbankmanagementsystem}
\newacronym{SQL}{SQL}{Structured Query Language}
\newacronym{MaRisk}{MaRisk}{Mindestanforderungen an das Risikomanagement}
\newacronym{LiqV}{LiqV}{Liquiditätsverordnung}

%--------------------------------------------------------------------------------------
% Glossareinträge
%--------------------------------------------------------------------------------------
\newglossaryentry{glos:bankrun}{
first=Bankenpanik\textsuperscript{GL},
name=Bankenpanik,
description={Eine Bankenpanik ist ein Ereignis, bei dem eine große Anzahl von Anlegern versucht, ihre Einlagern bei einer Bank abzuziehen. Der Grund kann zum Einen in der Veröffentlichung von schlechten Ergebnissen der Bank und damit einem Vertrauensverlust begründet sein, zum Anderen aber auch rein spekulativ sein. Für die Bank besteht die Gefahr der Insolvenz. Im Englischen spricht man von einem Bank Run.\seFootcite{vgl.}{S.1f}{Sch.11}}
}

\newglossaryentry{glos:sqlscript}{
first=SQLScript\textsuperscript{GL},
name=SQLScript,
description={SQLScript ist eine Erweiterung der Abfragesprache SQL und wird in der Datenbank von SAP HANA verwendet. Mit Hilfe von SQLScript lässt sich Anwendungslogik in die Datenbank auslagern. Dazu wurden unter Anderem Datentypen, Prozeduren und Kontrollstrukturen hinzugefügt.\seFootcite{vgl.}{S.9f}{SQLScript.11}}
}

%--------------------------------------------------------------------------------------
% Symbole
%--------------------------------------------------------------------------------------
\newglossaryentry{symb:pi}{
name=$\pi$,
description={Die Kreiszahl},
type=symbolslist,
sort=a
}