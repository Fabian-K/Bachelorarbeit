%--------------------------------------------------------------------------------------
% Trennungsregeln
%--------------------------------------------------------------------------------------
\hyphenation{prob-lem-los}

%--------------------------------------------------------------------------------------
% Abkürzungen
%--------------------------------------------------------------------------------------
\newacronym{HANA}{{HANA}}{SAP High Performance Analytic Appliance}
\newacronym{DBMS}{DBMS}{Datenbankmanagementsystem}
\newacronym{SQL}{SQL}{Structured Query Language}
\newacronym{MaRisk}{MaRisk}{Mindestanforderungen an das Risikomanagement}
\newacronym{LiqV}{LiqV}{Liquiditätsverordnung}
\newacronym{NGAP}{NGAP}{Next Generation ABAP Plattform}
\newacronym{ERP}{ERP}{Enterprise Resource Planning}
\newacronym{CRM}{CRM}{Customer Relationship Management}
\newacronym{LCR}{LCR}{Mindestliquiditätsquote (engl. Liqudiity Coverage Ration)}
\newacronym{NSFR}{NSFR}{Strukturelle Liquiditätsquote (engl. Net Stable Funding Ratio)}

\newacronym{LRM}{SAP~LRM}{SAP Liquidity Risk Management}


%--------------------------------------------------------------------------------------
% Glossareinträge
%--------------------------------------------------------------------------------------
\newglossaryentry{glos:bankrun}{
first=Bankenpanik\textsuperscript{GL},
name=Bankenpanik,
description={Eine Bankenpanik ist ein Ereignis, bei dem eine große Anzahl von Anlegern versucht, ihre Einlagern bei einer Bank abzuziehen. Der Grund kann zum Einen in der Veröffentlichung von schlechten Ergebnissen der Bank und damit einem Vertrauensverlust begründet sein, zum Anderen aber auch rein spekulativ sein. Für die Bank besteht die Gefahr der Insolvenz. Im Englischen spricht man von einem Bank Run.\seFootcite{vgl.}{S.1f}{Sch.11}}
}

\newglossaryentry{glos:sqlscript}{
first=SQLScript\textsuperscript{GL},
name=SQLScript,
description={SQLScript ist eine Erweiterung der Abfragesprache SQL und wird in der Datenbank von SAP HANA verwendet. Mit Hilfe von SQLScript lässt sich Anwendungslogik in die Datenbank auslagern. Dazu wurden unter Anderem Datentypen, Prozeduren und Kontrollstrukturen hinzugefügt.\seFootcite{vgl.}{S.9f}{SQLScript.11}}
}

\newglossaryentry{glos:ria}{
first=Rich Internet Application (RIA)\textsuperscript{GL},
name=Rich Internet Application (RIA),
description={Unter dem Begriff Rich Internet Application werden Webanwendungen bezeichnet, die in ihrer Funktionalität und ihrem Aussehen Desktopanwendungen ähneln. Erstmals eingeführt wurde der Begriff von Macromedia. Zwischen normalen Webanwendungen und RIA kann keine klare Grenze gezogen werden. Wichtiges Indiz für eine RIA ist der Einsatz von Technologien wie z.B. Adobe Flash, Adobe Air oder Microsoft Silverlight
\footnote{\seCite{vgl.}{S.32f}{DD.08} \seCite{und}{S.3f}{Pfe.09}\newline
Adobe Flash - http://www.adobe.com/products/flashplayer.html \newline
Adobe Air - http://www.adobe.com/products/air.html \newline
Microsoft Silverlight - http://www.microsoft.com/silverlight/} }
}

\newglossaryentry{glos:bydesign}{
first=SAP Business ByDesign\textsuperscript{GL},
name=SAP Business ByDesign,
description={SAP Business ByDesign ist eine Anwendung von SAP für mittelständige Unternehmen. Zu dem Funktionsunfang gehört sowohl ein \gls{ERP}- als auch eine \gls{CRM}-Lösung. Die Besonderheit von SAP Business ByDesign ist, dass es auf Servern bei SAP betrieben wird und Kunden die Anwendung mieten und über das Internet konsumieren.}
}

\newglossaryentry{glos:interbankenhandel}{
first=Interbankenhandel\textsuperscript{GL},
name=Interbankenhandel,
description={Interbankenhandel ist der Handel von Wertpapieren, Anlagen oder ähnlichem zwischen Banken. Synonym wird auch oft der Begriff Interbankenmarkt verwendet. Für die Zinssätze, mit denen Banken untereinander handeln, existieren anerkannte Referenzen, wie z.B. der LIBOR. (London Inter Bank Offered Rate). In Liquiditätsengpässen kann der Interbankenhandel eine wichtige Refinanzierungsrolle darstellen. Der Handel zwischen Banken hängt sehr stark von dem gegenseitigen Vertrauen ab.\seFootcite{vgl.}{S.382ff}{Lou.10}}
}

\newglossaryentry{glos:netFramework}{
first=.NET Framework\textsuperscript{GL},
name=.NET Framework,
description={ Das .NET Framework ist eine Plattform von Microsoft, mit der Anwendungen für das Betriebssystem Microsoft Windows erstellt und ausgeführt werden können. Die wichtigsten Komponenten sind Klassenbibliotheken, z.B. für die Entwicklung der Oberflächen, und die Common Language Runtime. Die Anwendungen können in verschiedenen Programmiersprachen geschrieben werden. Zu den unterstützten Sprachen zählen unter Anderem C++ und C\#. Der Quellcode wird dann in die programmiersprachenunabhängige Common Intermediate Language compiliert. Diese Zwischensprache kann dann von der Common Language Runtime ausgeführt werden.\seFootcite{vgl.}{S.145f}{Wil.10}}
}


%--------------------------------------------------------------------------------------
% Symbole
%--------------------------------------------------------------------------------------

%\newglossaryentry{symb:pi}{
%name=$\pi$,
%description={Die Kreiszahl},
%type=symbolslist,
%sort=a
%}