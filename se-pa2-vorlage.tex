% --------------------------------------------------------------------------------------------------------------
% Einlesen der .sty-Dateien
% --------------------------------------------------------------------------------------------------------------
%  se-pa1-input-styles.tex
%
%  Joerg Baumgart 01.08.2011
%
%  Zusammenfassung und Konfiguration wichtiger Styles f\"ur die 
%  Erzeugung von Seminar-, Projekt- und Bachelorarbeiten
%
%
\documentclass[12pt,BCOR=10mm,headinclude=on,footinclude=off,bibliography=totoc]{scrreprt}
\usepackage[T1]{fontenc}
\usepackage[utf8]{inputenc}
\usepackage[ngerman]{babel} % Deutsche Einstellungen
\usepackage{lmodern}

\usepackage{tikz} % Graphikpaket, das zu pdfLaTeX kompatibel ist
\usepackage{xkeyval} % Definition von Kommandos mit mehreren optionalen Argumenten
\usepackage{listings} % Formatierung von Programmlistings
\usepackage{graphicx} % Einbinden von Graphiken
\usepackage{ifthen}
\usepackage{color}
\usepackage{slashbox} % Diagonalen in Tabellenfeldern
\usepackage{framed} % Erzeugung schwarzer Linien am linken Rand zur Hervorhebung von Textteilen
\usepackage{mdframed}
\usepackage{caption} % Korrektes Setzen einer mehrzeiligen float-Unterschrift bei neu definierten float-Umgebungen
%\usepackage{floatrow}

% Es wird jeweils die sty-Datei importiert und entsprechende Konfigurationseinstellungen werden vorgenommen
%
\usepackage{sty/se-jb-scrpage2} % Formatierung der Kopf- und Fu{\ss}zeilen
\usepackage{sty/se-jb-footmisc}    % Fussnoten besser formatieren

\usepackage{sty/se-jb-glossaries} % Abk\"urzungsverzeichnis, Symbolverzeichnis, Glossar
   
\usepackage{sty/se-jb-floatrow}    % Definition und Konfiguration von float-Umgebungen (figure, table, die neue programm-Umgebung)
% Achtung: se-jb-varioref muss nach se-jb-floatrow importiert werden; 
% andernfalls ist der counter programm f\"ur die labelformat-Anweisung noch nicht definiert   
\usepackage{sty/se-jb-varioref}   % Definition von Querverweisen
\usepackage{sty/se-jb-chngcntr}   % Kapitelweise oder globale Nummerierung von Abbildungen etc.
   
\usepackage{sty/se-jb-listen} % Definition neuer, besser formatierter Listen
\usepackage{sty/se-jb-wa-kommandos} % neue Kommandos f\"ur Seminar-, Projekt- und Bachelorarbeiten


% --------------------------------------------------------------------------------------------------------------
% Individuelle Konfiguration des Dokumentes
% --------------------------------------------------------------------------------------------------------------
%  Individuelle Konfiguration einer Projektarbeit
%
%
%
%

%
% Literaturverzeichnis
% 
%\usepackage{se-jb-jurabib-theisen} % Literaturverzeichnis gem\"ass den Vorgaben von Theisen aufbauen

\usepackage{bibgerm}
\usepackage{url}
\usepackage{booktabs}
\usepackage{array}

\providecommand{\seCite}[3]{\ifthenelse{\equal{#2}{}}{#1 \cite{#3}}{#1 \cite[#2]{#3}}}


\providecommand{\seFootcite}[3]{\ifthenelse{\equal{#2}{}}{\footnote{#1 \cite{#3}}}{\footnote{#1 \cite[#2]{#3}}}}


% Weitere Optionseinstellungen f\"ur das Koma-Script
%
% Zwischen Abs\"atzen einen Abstand von 0.5 \baselineskip erzeugen
\KOMAoption{parskip}{full}
%
% Vergleiche Duden "Gliederung von Nummern, S.111" 
% DIN 5008 anschauen, wenn sie neu ver\"offentlicht wurde
\KOMAoption{numbers}{noendperiod}
%
%



%  Voreinstellungen f\"ur floats
%  Durch die verwendeten Parameter wird die Wahrscheinlichkeit deutlich kleiner, 
%  dass Gleitobjekte (z. B. Abbildungen) ans Ende des Dokumentes verschoben 
%  werden; 
%  Achtung: clearpage erzwingt die Ausgabe von Gleitobjekten
%
\renewcommand{\topfraction}{1}  % Gleitobjekte d\"urfen eine Seite zu 100% belegen 
\renewcommand{\bottomfraction}{1} % Entsprechender Wert f\"ur den unteren Teil der Seite
\renewcommand{\textfraction}{0} % Eine Seite darf auch ohne Fliesstext existieren
%%%\renewcommand{\floatpagefraction}{1} % Bedeutung unklar, daher keine Ver\"anderung des Vorgabewertes 
                                                                        % von 0.5; eventuell bringt ein \"Anderung auf 1 etwas, wenn 
                                                                         % Probleme mit floats auftreten
                                                                         
                                                                         
                                                                         
% Konfiguration von Programm-Listings
% 
% Achtung: hier gibt es nahezu beliebig viele weitere Konfigurationm\"oglichkeiten; vgl. Paketdokumentation
%
\lstset{language=[R/3 6.10]ABAP,basicstyle=\ttfamily,keywordstyle=\color{blue},captionpos=b,aboveskip=0mm,belowskip=0mm,
          xleftmargin=0em}        


% Actionscrupt Highlightung


\definecolor{purple}{rgb}{0.65, 0.12, 0.82}
\definecolor{flexred}{rgb}{0.65, 0.01, 0.01}
\definecolor{flexgreen}{rgb}{0, 0.6, 0}
\definecolor{flexgray}{rgb}{0.25, 0.37, 0.75}
\definecolor{flexblue}{rgb}{0, 0.2, 1}
\definecolor{flexfunction}{rgb}{0.2, 0.6, 0.4}
\definecolor{flexvar}{rgb}{0.4, 0.6, 0.8}
\definecolor{black}{rgb}{0, 0, 0}



\lstdefinelanguage{ActionScript} {
   basicstyle=\ttfamily\scriptsize,
   sensitive=true,
   %morecomment=[l][\color{flexgreen}\ttfamily]{//},
   %morecomment=[s][\color{flexgreen}\ttfamily]{/*}{*/},
   %morecomment=[s][\color{flexgray}\ttfamily]{/**}{*/},
   morestring=[b]",
   morestring=[d]/,
   morecomment=[l][\color{flexgreen}\ttfamily]{//},
   morecomment=[s][\color{flexgreen}\ttfamily]{/*}{*/},
   morecomment=[s][\color{flexgray}\ttfamily]{/**}{*/},
   morecomment=[n][\color{flexblue}\ttfamily]{<}{>},
   stringstyle=\color{flexred}\textbf,
   commentstyle=\color{flexgreen},
   showstringspaces=false,
   numberstyle=\scriptsize,
   numberblanklines=true,
   showspaces=false,
   breaklines=true,
   showtabs=false,
   emph =
   {[1]
      class, package, interface
   },
   emphstyle={[1]\color{purple}\textbf},
   emph =
   {[2]
      internal, public, protected, private,
      super, this, import, new, extends, implements,
      void, true, false, as
   },
   emphstyle={[2]\color{flexblue}\textbf},
   emph =
   {[3]
      function
   },
   emphstyle={[3]\color{flexfunction}\textbf},
   emph =
   {[4]
      var
   },
   emphstyle={[4]\color{flexvar}\textbf}
}




%
% Grundkonfiguration der Abs\"ande zwischen den Items der maximal f\"unf Verschachtelungsebenen der 
% neuen Listenumgebungen
%                                                                             
% Initialisierung der Abst\"ande zwischen den items f\"ur seList; Grundeinheit: 0.5\baselineskip; siehe se-jb-listen
\seSetlistbaselineskip{1}{0.75}{0.75}{0.75}{0.75}
% Initialisierung der Abst\"ande zwischen den items f\"ur seToplist; Grundeinheit: 0.5\baselineskip; siehe se-jb-listen
\seSettoplistbaselineskip{1}{0.75}{0.75}{0.75}{0.75}     


%
%  Konfiguration der verschiedenen Verzeichnisse
%
%  abstandEintrag: Wert wird mit \baselineskip multipliziert
%

%
%  Abbildungsverzeichnis
%
\seKonfigurationAbb[
%verzeichnisname=Abbildungsverzeichnis,
labeltextLinks=, % kein Text links;
%labeltextRechts=:,
labelbreite=1cm,
%labeleinzug=1cm,
%abstandEintrag=1,
newpage=ja,
%pnumwidth=20mm,
%dotsep=1000,
%tocrmarg=4.5cm,
%abstandVerzeichnis=-1mm
]

%
% LIstingverzeichnis
%
\seKonfigurationPrg[
%verzeichnisname=Listing-Verzeichnis,
labeltextLinks=,
%labeltextRechts=:,
labelbreite=1cm,
%labeleinzug=2cm,
%abstandEintrag=1,
newpage=ja,
%%pnumwidth=20mm,
%dotsep=1000,
%tocrmarg=4.5cm,
%abstandVerzeichnis=-10mm
]

%
% Tabellenverzeichnis
%
\seKonfigurationTab[
%verzeichnisname=Liste der Tabellen,
labeltextLinks=,
%labeltextRechts=:,
labelbreite=1cm,
%labeleinzug=0.5cm,
%abstandEintrag=1,
newpage=ja,
%pnumwidth=20mm,
%dotsep=1000,
%tocrmarg=4.5cm,
%abstandVerzeichnis=-10mm
]

%
% Abk\"urzungsverzeichnis
%
\seKonfigurationAbk[
%verzeichnisname=Liste der Abk\"urzungen,
%labelbreite=3cm,
%labeleinzug=0.5cm,
%abstandEintrag=1,
%newpage=ja,
%abstandVerzeichnis=-10mm
]

%
% Symbolverzeichnis
% 
\seKonfigurationSym[
%verzeichnisname=Liste der Symbole,
%labelbreite=4cm,
%labeleinzug=3.5cm,
%abstandEintrag=1,
newpage=ja,
%abstandVerzeichnis=-10mm
]

%
% Glossar
%
\seKonfigurationGlo[
%verzeichnisname=Glossar,
%abstandEintrag=0,
]



% (eventuelle) Neudefinition f\"ur die Unter-/\"Uberschriften von Abbildungen, Tabellen und Listings
%
%
%\renewcommand{\seCaptionNameAbbildung}{Abb.}
%\renewcommand{\seCaptionNameTabelle}{Tab.}
%\renewcommand{\seCaptionNameProgramm}{Prg.}


% % (eventuelle) Neudefinition f\"ur Querverweise innerhalb des Textes
%
%
%
%\renewcommand{\seQuerverweisSeite}{Seite}
%\renewcommand{\seQuerverweisAbbildung}{Abb.}
%\renewcommand{\seQuerverweisTabelle}{Tab.}
%\renewcommand{\seQuerverweisProgramm}{Prg.}
%\renewcommand{\seQuerverweisKapitel}{Kap.}
%\renewcommand{\seQuerverweisGleichung}{Gl.}

% Kommandos, die direkt nach \begin{document} ausgef\"uhrt werden m\"ussen
%
%
%
\AtBeginDocument{%
\renewcommand{\listfigurename}{\seAbbildungenVerzeichnisname}
\renewcommand{\listtablename}{\seTabellenVerzeichnisname}
\renewcommand{\figurename}{\seCaptionNameAbbildung}
\renewcommand{\tablename}{\seCaptionNameTabelle}
\pagenumbering{roman}
}
                                                              
                                                                         

% --------------------------------------------------------------------------------------------------------------
% Individuelle Definition von Abk\"urzungen, Symbolen und eventuell Glossareintr\"agen
% --------------------------------------------------------------------------------------------------------------
%--------------------------------------------------------------------------------------
% Trennungsregeln
%--------------------------------------------------------------------------------------
\hyphenation{prob-lem-los}

%--------------------------------------------------------------------------------------
% Abkürzungen
%--------------------------------------------------------------------------------------
\newacronym{HANA}{{HANA}}{SAP High Performance Analytic Appliance}
\newacronym{DBMS}{DBMS}{Datenbankmanagementsystem}
\newacronym{SQL}{SQL}{Structured Query Language}
\newacronym{MaRisk}{MaRisk}{Mindestanforderungen an das Risikomanagement}
\newacronym{LiqV}{LiqV}{Liquiditätsverordnung}

%--------------------------------------------------------------------------------------
% Glossareinträge
%--------------------------------------------------------------------------------------
\newglossaryentry{glos:bankrun}{
first=Bankenpanik\textsuperscript{GL},
name=Bankenpanik,
description={Eine Bankenpanik ist ein Ereignis, bei dem eine große Anzahl von Anlegern versucht, ihre Einlagern bei einer Bank abzuziehen. Der Grund kann zum Einen in der Veröffentlichung von schlechten Ergebnissen der Bank und damit einem Vertrauensverlust begründet sein, zum Anderen aber auch rein spekulativ sein. Für die Bank besteht die Gefahr der Insolvenz. Im Englischen spricht man von einem Bank Run.\seFootcite{vgl.}{S.1f}{Sch.11}}
}

\newglossaryentry{glos:sqlscript}{
first=SQLScript\textsuperscript{GL},
name=SQLScript,
description={SQLScript ist eine Erweiterung der Abfragesprache SQL und wird in der Datenbank von SAP HANA verwendet. Mit Hilfe von SQLScript lässt sich Anwendungslogik in die Datenbank auslagern. Dazu wurden unter Anderem Datentypen, Prozeduren und Kontrollstrukturen hinzugefügt.\seFootcite{vgl.}{S.9f}{SQLScript.11}}
}

%--------------------------------------------------------------------------------------
% Symbole
%--------------------------------------------------------------------------------------
\newglossaryentry{symb:pi}{
name=$\pi$,
description={Die Kreiszahl},
type=symbolslist,
sort=a
} 

\begin{document}

\newcommand{\version}{0.1}

% --------------------------------------------------------------------------------------------------------------
% Erzeugung des Titelblatts
% --------------------------------------------------------------------------------------------------------------
\seTitelblattZweiteProjektarbeit[
firmenlogo=images/sap-logo,
firmenlogoDeltaX=-13,
firmenlogoDeltaY=-10,
thema=Entwicklung einer Zwischenschicht für die Nutzung weiterer Anwendungen in Verbindung mit der Berechnungskomponente des Liquidity Risk Managements,
verfasser=Fabian Kajzar,
matrikelnummer=428094,
kurs=WWI\,09\,SW\,B,
firma=SAP AG,
abteilung=Application Strategic Innovation - HPA,
studiengangsleiter=Prof. Dr.-Ing. J\"org Baumgart,
wissenschaftlicherBetreuerName=Prof. Dr. Hans-Henning Pagnia,
wissenschaftlicherBetreuerEmail=hans-henning.pagnia@dhbw-mannheim.de,
wissenschaftlicherBetreuerTelefon=0621 4105-1131,
firmenbetreuerName=Jens Mett,
firmenbetreuerEmail=jens.mett@sap.com,
firmenbetreuerTelefon=06227 7-61785,
bearbeitungszeitraumVon=13. Februar 2012,
bearbeitungszeitraumBis=4. Mai 2012,
sperrvermerk=nein
]

\pagenumbering{Roman}
\setcounter{page}{0}

% --------------------------------------------------------------------------------------------------------------
% Erzeugung der Kurzfassung; Verfasser, Firma und Thema werden automatisch \"ubernommen
% --------------------------------------------------------------------------------------------------------------
\seKurzfassung{}

\glsresetall

% Beispiel f\"ur ein Kapitel, dass vor dem Einleitungskapitel kommt, z. B. ein Vorwort oder eine Danksagung
%\seKapitelVorEinleitung{Vorwort}

% --------------------------------------------------------------------------------------------------------------
% Ausgabe des Inhaltsverzeichnisses
% --------------------------------------------------------------------------------------------------------------
\seInhaltsverzeichnis[
einrueckung=ja,
gliederungsebenen=4
]

% --------------------------------------------------------------------------------------------------------------
% Ausgabe der verschiedenen Verzeichnisse
% --------------------------------------------------------------------------------------------------------------
\seVerzeichnisse[gliederungsebene=section,imInhaltsverzeichnis=ja]{abk}{sym}{abb}{tab}{prg}


% --------------------------------------------------------------------------------------------------------------
% Vorbereitung für Inhalt der Arbeit
% --------------------------------------------------------------------------------------------------------------
\pagenumbering{arabic}

% --------------------------------------------------------------------------------------------------------------
% Inhalt der Arbeit
% --------------------------------------------------------------------------------------------------------------
\chapter{Testkapitel}

\section{Bild}

\begin{figure}[h]
\centering
\setlength{\unitlength}{1mm}
\includegraphics[width=15cm]{images/sap-logo.png}
\caption[Umfang der Cloud-Angebote als Venn-Diagramm]{Umfang der Cloud-Angebote als Venn-Diagramm, eigene Darstellung\label{fig:grundlagen:venn}}
\end{figure}

\section{Tabelle}

\begin{table}[htbp]%
\centering%
\begin{tabular}{| p{3cm} | p{5cm} |}
\hline
Speicher & Zugriffszeit \\
\hline
\hline
L1 Cache & ca. 4 CPU-Zyklen  \\ \hline
Hauptspeicher & 100 - 400 CPU-Zyklen \\ \hline
Festplatte & 1.000.000 CPU-Zyklen \\ \hline
\end{tabular} 
\caption[Vergleich der Zugriffszeiten von Speicher]{Vergleich der Zugriffszeiten von Speicher\label{table:zugriff}\protect \footnotemark}
\end{table}
\footnotetext{\seCite{vgl.}{S.12}{newDB-bluebook}}

\section{Glossar}
Das hier ist ein Eintrag im Glossar: \gls{glos:test}

\section{Abkürzung}
Lorem ipsum dolor sit amet, consectetur adipisicing elit, sed do eiusmod tempor incididunt ut labore et dolore magna aliqua. Ut enim ad minim veniam, quis nostrud exercitation ullamco laboris nisi ut aliquip ex ea commodo consequat.

\section{Listing}
Ein Beispielprogramm mit ABAP Syntax-Highlighting ist in \vref{hello} zu sehen.

\begin{programm}[htbp]
\begin{lstlisting}
REPORT TEST.

DATA LASTNAME TYPE STRING.

WRITE 'Hello World'.
REPLACE 'A' WITH 'B' INTO LASTNAME.

SELECT * FROM FLIGHTINFO WHERE CLASS = 'Y' OR CLASS = 'C'.
\end{lstlisting}
\caption{Die Klasse \texttt{HelloDHBW}\label{hello}}
\end{programm}

\section{Zitat}
Lorem ipsum dolor sit amet, consectetur adipisicing elit, sed do eiusmod tempor incididunt ut labore et dolore magna aliqua. Ut enim ad minim veniam, quis nostrud exercitation ullamco laboris nisi ut aliquip ex ea commodo consequat.\footnote{\seCite{vgl.}{S.7}{cc-hmd} \seCite{und}{S.4}{cc-iif}}

\section{Formel}
Lorem ipsum dolor sit amet, consectetur adipisicing elit, sed do eiusmod tempor incididunt ut labore et dolore magna aliqua. Ut enim ad minim veniam, quis nostrud exercitation ullamco laboris nisi ut aliquip ex ea commodo consequat.

\section{Footnote}
Lorem ipsum dolor sit amet, consectetur adipisicing elit, sed do eiusmod tempor incididunt ut labore et dolore magna aliqua.\footnote{eine ganz normale Fußnote, reiner Text...} Ut enim ad minim veniam, quis nostrud exercitation ullamco laboris nisi ut aliquip ex ea commodo consequat.

% --------------------------------------------------------------------------------------------------------------
% Anhang der Arbeit
% --------------------------------------------------------------------------------------------------------------
\seAppendix{}

\pagenumbering{Roman}
\setcounter{page}{7}

\chapter{Anhang}

Inhalt des Anhangs

% --------------------------------------------------------------------------------------------------------------
%  Erzeugung eines Glossars
% --------------------------------------------------------------------------------------------------------------
\newpage
\sePrintGlossary{}


% --------------------------------------------------------------------------------------------------------------
% Literaturverzeichnisses
% --------------------------------------------------------------------------------------------------------------
\sePrintBibliography{}

\bibliographystyle{mystyle}
\seBibliography{wa}


% --------------------------------------------------------------------------------------------------------------
% Erzeugung der ehrenw\"ortlichen Erkl\"arung
% --------------------------------------------------------------------------------------------------------------
\seEhrenwoertlicheErklaerung{}

\end{document}
