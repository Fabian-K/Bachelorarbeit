%
% Einlesen der .sty-Dateien
%
%  se-pa1-input-styles.tex
%
%  Joerg Baumgart 01.08.2011
%
%  Zusammenfassung und Konfiguration wichtiger Styles f\"ur die 
%  Erzeugung von Seminar-, Projekt- und Bachelorarbeiten
%
%
\documentclass[12pt,BCOR=10mm,headinclude=on,footinclude=off,bibliography=totoc]{scrreprt}
\usepackage[T1]{fontenc}
\usepackage[utf8]{inputenc}
\usepackage[ngerman]{babel} % Deutsche Einstellungen
\usepackage{lmodern}

\usepackage{tikz} % Graphikpaket, das zu pdfLaTeX kompatibel ist
\usepackage{xkeyval} % Definition von Kommandos mit mehreren optionalen Argumenten
\usepackage{listings} % Formatierung von Programmlistings
\usepackage{graphicx} % Einbinden von Graphiken
\usepackage{ifthen}
\usepackage{color}
\usepackage{slashbox} % Diagonalen in Tabellenfeldern
\usepackage{framed} % Erzeugung schwarzer Linien am linken Rand zur Hervorhebung von Textteilen
\usepackage{mdframed}
\usepackage{caption} % Korrektes Setzen einer mehrzeiligen float-Unterschrift bei neu definierten float-Umgebungen
%\usepackage{floatrow}

% Es wird jeweils die sty-Datei importiert und entsprechende Konfigurationseinstellungen werden vorgenommen
%
\usepackage{sty/se-jb-scrpage2} % Formatierung der Kopf- und Fu{\ss}zeilen
\usepackage{sty/se-jb-footmisc}    % Fussnoten besser formatieren

\usepackage{sty/se-jb-glossaries} % Abk\"urzungsverzeichnis, Symbolverzeichnis, Glossar
   
\usepackage{sty/se-jb-floatrow}    % Definition und Konfiguration von float-Umgebungen (figure, table, die neue programm-Umgebung)
% Achtung: se-jb-varioref muss nach se-jb-floatrow importiert werden; 
% andernfalls ist der counter programm f\"ur die labelformat-Anweisung noch nicht definiert   
\usepackage{sty/se-jb-varioref}   % Definition von Querverweisen
\usepackage{sty/se-jb-chngcntr}   % Kapitelweise oder globale Nummerierung von Abbildungen etc.
   
\usepackage{sty/se-jb-listen} % Definition neuer, besser formatierter Listen
\usepackage{sty/se-jb-wa-kommandos} % neue Kommandos f\"ur Seminar-, Projekt- und Bachelorarbeiten


%
% Individuelle Konfiguration des Dokumentes
%
%  Individuelle Konfiguration einer Projektarbeit
%
%
%
%

%
% Literaturverzeichnis
% 
%\usepackage{se-jb-jurabib-theisen} % Literaturverzeichnis gem\"ass den Vorgaben von Theisen aufbauen

\usepackage{bibgerm}
\usepackage{url}
\usepackage{booktabs}
\usepackage{array}

\providecommand{\seCite}[3]{\ifthenelse{\equal{#2}{}}{#1 \cite{#3}}{#1 \cite[#2]{#3}}}


\providecommand{\seFootcite}[3]{\ifthenelse{\equal{#2}{}}{\footnote{#1 \cite{#3}}}{\footnote{#1 \cite[#2]{#3}}}}


% Weitere Optionseinstellungen f\"ur das Koma-Script
%
% Zwischen Abs\"atzen einen Abstand von 0.5 \baselineskip erzeugen
\KOMAoption{parskip}{full}
%
% Vergleiche Duden "Gliederung von Nummern, S.111" 
% DIN 5008 anschauen, wenn sie neu ver\"offentlicht wurde
\KOMAoption{numbers}{noendperiod}
%
%



%  Voreinstellungen f\"ur floats
%  Durch die verwendeten Parameter wird die Wahrscheinlichkeit deutlich kleiner, 
%  dass Gleitobjekte (z. B. Abbildungen) ans Ende des Dokumentes verschoben 
%  werden; 
%  Achtung: clearpage erzwingt die Ausgabe von Gleitobjekten
%
\renewcommand{\topfraction}{1}  % Gleitobjekte d\"urfen eine Seite zu 100% belegen 
\renewcommand{\bottomfraction}{1} % Entsprechender Wert f\"ur den unteren Teil der Seite
\renewcommand{\textfraction}{0} % Eine Seite darf auch ohne Fliesstext existieren
%%%\renewcommand{\floatpagefraction}{1} % Bedeutung unklar, daher keine Ver\"anderung des Vorgabewertes 
                                                                        % von 0.5; eventuell bringt ein \"Anderung auf 1 etwas, wenn 
                                                                         % Probleme mit floats auftreten
                                                                         
                                                                         
                                                                         
% Konfiguration von Programm-Listings
% 
% Achtung: hier gibt es nahezu beliebig viele weitere Konfigurationm\"oglichkeiten; vgl. Paketdokumentation
%
\lstset{language=[R/3 6.10]ABAP,basicstyle=\ttfamily,keywordstyle=\color{blue},captionpos=b,aboveskip=0mm,belowskip=0mm,
          xleftmargin=0em}        


% Actionscrupt Highlightung


\definecolor{purple}{rgb}{0.65, 0.12, 0.82}
\definecolor{flexred}{rgb}{0.65, 0.01, 0.01}
\definecolor{flexgreen}{rgb}{0, 0.6, 0}
\definecolor{flexgray}{rgb}{0.25, 0.37, 0.75}
\definecolor{flexblue}{rgb}{0, 0.2, 1}
\definecolor{flexfunction}{rgb}{0.2, 0.6, 0.4}
\definecolor{flexvar}{rgb}{0.4, 0.6, 0.8}
\definecolor{black}{rgb}{0, 0, 0}



\lstdefinelanguage{ActionScript} {
   basicstyle=\ttfamily\scriptsize,
   sensitive=true,
   %morecomment=[l][\color{flexgreen}\ttfamily]{//},
   %morecomment=[s][\color{flexgreen}\ttfamily]{/*}{*/},
   %morecomment=[s][\color{flexgray}\ttfamily]{/**}{*/},
   morestring=[b]",
   morestring=[d]/,
   morecomment=[l][\color{flexgreen}\ttfamily]{//},
   morecomment=[s][\color{flexgreen}\ttfamily]{/*}{*/},
   morecomment=[s][\color{flexgray}\ttfamily]{/**}{*/},
   morecomment=[n][\color{flexblue}\ttfamily]{<}{>},
   stringstyle=\color{flexred}\textbf,
   commentstyle=\color{flexgreen},
   showstringspaces=false,
   numberstyle=\scriptsize,
   numberblanklines=true,
   showspaces=false,
   breaklines=true,
   showtabs=false,
   emph =
   {[1]
      class, package, interface
   },
   emphstyle={[1]\color{purple}\textbf},
   emph =
   {[2]
      internal, public, protected, private,
      super, this, import, new, extends, implements,
      void, true, false, as
   },
   emphstyle={[2]\color{flexblue}\textbf},
   emph =
   {[3]
      function
   },
   emphstyle={[3]\color{flexfunction}\textbf},
   emph =
   {[4]
      var
   },
   emphstyle={[4]\color{flexvar}\textbf}
}




%
% Grundkonfiguration der Abs\"ande zwischen den Items der maximal f\"unf Verschachtelungsebenen der 
% neuen Listenumgebungen
%                                                                             
% Initialisierung der Abst\"ande zwischen den items f\"ur seList; Grundeinheit: 0.5\baselineskip; siehe se-jb-listen
\seSetlistbaselineskip{1}{0.75}{0.75}{0.75}{0.75}
% Initialisierung der Abst\"ande zwischen den items f\"ur seToplist; Grundeinheit: 0.5\baselineskip; siehe se-jb-listen
\seSettoplistbaselineskip{1}{0.75}{0.75}{0.75}{0.75}     


%
%  Konfiguration der verschiedenen Verzeichnisse
%
%  abstandEintrag: Wert wird mit \baselineskip multipliziert
%

%
%  Abbildungsverzeichnis
%
\seKonfigurationAbb[
%verzeichnisname=Abbildungsverzeichnis,
labeltextLinks=, % kein Text links;
%labeltextRechts=:,
labelbreite=1cm,
%labeleinzug=1cm,
%abstandEintrag=1,
newpage=ja,
%pnumwidth=20mm,
%dotsep=1000,
%tocrmarg=4.5cm,
%abstandVerzeichnis=-1mm
]

%
% LIstingverzeichnis
%
\seKonfigurationPrg[
%verzeichnisname=Listing-Verzeichnis,
labeltextLinks=,
%labeltextRechts=:,
labelbreite=1cm,
%labeleinzug=2cm,
%abstandEintrag=1,
newpage=ja,
%%pnumwidth=20mm,
%dotsep=1000,
%tocrmarg=4.5cm,
%abstandVerzeichnis=-10mm
]

%
% Tabellenverzeichnis
%
\seKonfigurationTab[
%verzeichnisname=Liste der Tabellen,
labeltextLinks=,
%labeltextRechts=:,
labelbreite=1cm,
%labeleinzug=0.5cm,
%abstandEintrag=1,
newpage=ja,
%pnumwidth=20mm,
%dotsep=1000,
%tocrmarg=4.5cm,
%abstandVerzeichnis=-10mm
]

%
% Abk\"urzungsverzeichnis
%
\seKonfigurationAbk[
%verzeichnisname=Liste der Abk\"urzungen,
%labelbreite=3cm,
%labeleinzug=0.5cm,
%abstandEintrag=1,
%newpage=ja,
%abstandVerzeichnis=-10mm
]

%
% Symbolverzeichnis
% 
\seKonfigurationSym[
%verzeichnisname=Liste der Symbole,
%labelbreite=4cm,
%labeleinzug=3.5cm,
%abstandEintrag=1,
newpage=ja,
%abstandVerzeichnis=-10mm
]

%
% Glossar
%
\seKonfigurationGlo[
%verzeichnisname=Glossar,
%abstandEintrag=0,
]



% (eventuelle) Neudefinition f\"ur die Unter-/\"Uberschriften von Abbildungen, Tabellen und Listings
%
%
%\renewcommand{\seCaptionNameAbbildung}{Abb.}
%\renewcommand{\seCaptionNameTabelle}{Tab.}
%\renewcommand{\seCaptionNameProgramm}{Prg.}


% % (eventuelle) Neudefinition f\"ur Querverweise innerhalb des Textes
%
%
%
%\renewcommand{\seQuerverweisSeite}{Seite}
%\renewcommand{\seQuerverweisAbbildung}{Abb.}
%\renewcommand{\seQuerverweisTabelle}{Tab.}
%\renewcommand{\seQuerverweisProgramm}{Prg.}
%\renewcommand{\seQuerverweisKapitel}{Kap.}
%\renewcommand{\seQuerverweisGleichung}{Gl.}

% Kommandos, die direkt nach \begin{document} ausgef\"uhrt werden m\"ussen
%
%
%
\AtBeginDocument{%
\renewcommand{\listfigurename}{\seAbbildungenVerzeichnisname}
\renewcommand{\listtablename}{\seTabellenVerzeichnisname}
\renewcommand{\figurename}{\seCaptionNameAbbildung}
\renewcommand{\tablename}{\seCaptionNameTabelle}
\pagenumbering{roman}
}
                                                              
                                                                         

%
% Individuelle Definition von Abk\"urzungen, Symbolen und eventuell Glossareintr\"agen
%
%--------------------------------------------------------------------------------------
% Trennungsregeln
%--------------------------------------------------------------------------------------
\hyphenation{prob-lem-los}

%--------------------------------------------------------------------------------------
% Abkürzungen
%--------------------------------------------------------------------------------------
\newacronym{HANA}{{HANA}}{SAP High Performance Analytic Appliance}
\newacronym{DBMS}{DBMS}{Datenbankmanagementsystem}
\newacronym{SQL}{SQL}{Structured Query Language}
\newacronym{MaRisk}{MaRisk}{Mindestanforderungen an das Risikomanagement}
\newacronym{LiqV}{LiqV}{Liquiditätsverordnung}

%--------------------------------------------------------------------------------------
% Glossareinträge
%--------------------------------------------------------------------------------------
\newglossaryentry{glos:bankrun}{
first=Bankenpanik\textsuperscript{GL},
name=Bankenpanik,
description={Eine Bankenpanik ist ein Ereignis, bei dem eine große Anzahl von Anlegern versucht, ihre Einlagern bei einer Bank abzuziehen. Der Grund kann zum Einen in der Veröffentlichung von schlechten Ergebnissen der Bank und damit einem Vertrauensverlust begründet sein, zum Anderen aber auch rein spekulativ sein. Für die Bank besteht die Gefahr der Insolvenz. Im Englischen spricht man von einem Bank Run.\seFootcite{vgl.}{S.1f}{Sch.11}}
}

\newglossaryentry{glos:sqlscript}{
first=SQLScript\textsuperscript{GL},
name=SQLScript,
description={SQLScript ist eine Erweiterung der Abfragesprache SQL und wird in der Datenbank von SAP HANA verwendet. Mit Hilfe von SQLScript lässt sich Anwendungslogik in die Datenbank auslagern. Dazu wurden unter Anderem Datentypen, Prozeduren und Kontrollstrukturen hinzugefügt.\seFootcite{vgl.}{S.9f}{SQLScript.11}}
}

%--------------------------------------------------------------------------------------
% Symbole
%--------------------------------------------------------------------------------------
\newglossaryentry{symb:pi}{
name=$\pi$,
description={Die Kreiszahl},
type=symbolslist,
sort=a
} 

\usepackage{bibgerm}

\begin{document}

\newcommand{\version}{0.93}

% Erzeugung des Titelblatts
%
%
%
\seTitelblattZweiteProjektarbeit[
%hilfslinien=ja,
%dhbwlogoSkalierung=0.5,
%dhbwlogoDeltaX=2.4,
%dhbwlogoDeltaY=-10,
firmenlogo=firmenlogo,
firmenlogoSkalierung=0.5,
firmenlogoDeltaX=0,
firmenlogoDeltaY=0,
thema=\LaTeX-Layoutvorlage zur Anfertigung der zweiten Projektarbeit\\(Version \version),
verfasser=J\"org Baumgart,
%verfasserin=,
matrikelnummer=9999999,
kurs=WWI\,09\,SW\,B,
firma=Ausbildungsfirma,
abteilung=Wirtschafsinformatik/Softwaremethodik,
%studiengangsleiterin=,
studiengangsleiter=Prof. Dr.-Ing. J\"org Baumgart,
wissenschaftlicheBetreuerinName=Dr. Melanie Mustermann,
wissenschaftlicheBetreuerinEmail=melanie.mustermann@musterfirma.de,
wissenschaftlicheBetreuerinTelefon=0621/999999,
%wissenschaftlicherBetreuerName=,
%wissenschaftlicherBetreuerEmail=,
%wissenschaftlicherBetreuerTelefon=,
firmenbetreuerinName=Dipl.-Ing. Ariane Meistermann,
firmenbetreuerinEmail=a.meistermann@andere-musterfirma.de,
firmenbetreuerinTelefon=06151/88888,
%firmenbetreuerName=,
%firmenbetreuerEmail=,
%firmenbetreuerTelefon=,
bearbeitungszeitraumVon=22. Juli 2011,
bearbeitungszeitraumBis=27. Oktober 2011,
sperrvermerk=ja,
%sperrvermerkAlternativerText={Ein (inhaltlich) alternativer Text darf nur verwendet werden, wenn er vom Studiengangsleiter genehmigt wurde!}
]



% Erzeugung der Kurzfassung; Verfasser, Firma und Thema werden automatisch \"ubernommen
%
% Der optionale Parameter kann verwendet werden, um f\"ur das Thema der Arbeit eine 
% andere Formatierung vorzunehmen; das sollte in der Regel nicht erforderlich sein;
% ausserdem besteht die Gefahr inkonsistenter Titel auf dem Titelblatt und in der 
% Kurzfassung
%
%\seKurzfassung{} % dieses Kommando sollte standardm\"assig verwendet werden
\seKurzfassung[\LaTeX-Layoutvorlage zur Anfertigung der zweiten Projektarbeit (Version \version)]{}

% Beispiel f\"ur ein Kapitel, dass vor dem Einleitungskapitel kommt, z. B. ein Vorwort oder eine Danksagung
\seKapitelVorEinleitung{Vorwort}

Diese \LaTeX{}-Vorlage soll es erm\"oglichen, ohne tiefergehende \LaTeX-Kenntnisse 
eine \seArbeit{} erstellen zu k\"onnen, die die Empfehlungen und Vorgaben aus dem 
Dokument \textbf{Empfehlungen und Hinweise zur Anfertigung der zweiten Projektarbeit (Version 0.92)}
erf\"ullt. 

Das vorliegende Dokument besitzt die Version \version, da die zugeh\"origen sty-Dateien 
nach und nach erg\"anzt werden, um f\"ur Seminar-, Projekt- und Bachelorarbeiten neue 
Kommandos f\"ur eine effiziente Dokumenterstellung anbieten zu k\"onnen. 
Neue Versionen werden abw\"artskompatibel sein. 



% Ausgabe des Inhaltsverzeichnisses
%
%
\seInhaltsverzeichnis[%
einrueckung=ja,
gliederungsebenen=4
]



% Ausgabe der verschiedenen Verzeichnisse
% abk: Abk\"urzungsverzeichnis
% sym: Symbolverzeichnis
% abb: Abbildungsverzeichnis
% tab: Tabellenverzeichnis
% prg: Listingverzeichnis
%
%
% Achtung: Abk\"urzungs- und Symbolverzeichnis werden nur ausgegeben, wenn mindest ein Symbol bzw. 
%                mindestens eine Abk\"urzung in der Arbeit verwendet wurden
%
%
% gliederungsebene:
% -- section: die Verzeichnisse werden einem Kapitel "Verzeichnisse" untergliedert
% -- chapter: die Verzeichnisse sind jeweils eigene Kapitel
% imInhaltsverzeichnis: ja/nein -- Sollen die Verzeichnisse im Inhaltsverzeichnis enthalten sein?
\seVerzeichnisse[gliederungsebene=section,imInhaltsverzeichnis=ja]{abk}{sym}{abb}{tab}{prg}






% Erstes eigentliches Kapitel der Arbeit; typischerweise das Einleitungskapitel;
% hier muss wieder auf die Nummierung mit arabischen Seitenzahlen umgestellt werden
%
\chapter{Einleitung}\pagenumbering{arabic}


test \seFootcite{vgl.}{S.5}{test}

% Erstes Hauptkapitel der Arbeit 
%
%
%
\chapter{Der formale Aufbau einer Projektarbeit}
% Mit markright kann eine verk\"urzte Version der \"Uberschrift f\"ur den Seitenkopf generiert werden
%
%
%\markright{Formaler Aufbau}




% Anhang der Arbeit
% 
%
\seAppendix{}
\chapter{Einige wichtige \LaTeX{}-Kommandos}



%  Testdatei f\"ur die Erzeugung von Literaturreferenzen, die den Regeln von Rene Theisen 
%  (Wissenschaftliches Arbeiten, 2009) folgen
%
%
%
\section{Kommandos f\"ur die Erzeugung von Literaturverweisen}

Test

%
% Ein kleiner Text, um Abk\"urzungen, Symbole und Glossareintr\"age zu testen
%
%
\section{Kommandos f\"ur die Erzeugung von Abk\"urzungen, Symbolen und Glos\-sar\-eint\-r\"a\-gen}

asd

\section{Abbildungen, Tabellen und Programmlistings\label{gleitobjekte}}

asd

\section{Die Definition und Anwendung von zwei neuen Listenumgebungen}
asd


\chapter{Hinweise zur Installation und \"Ubersetzung}

\input{se-hinweise-installation}

%
%  Erzeugung eines Glossars
%
% Achtung: Das Glossar wird nur ausgegeben, wenn mindestens ein Eintrag in der Arbeit 
%                definiert wurde
%
%
\newpage
\sePrintGlossary{}


%
% Literaturverzeichnisses
%
%\newpage
\sePrintBibliography{}

%  Erzeugung von Eintr\"agen im Literaturverzeichnis
%
%  Achtung: in einer Projektarbeit darf da \nocite-Kommando nicht verwendet werden,
%                 da es einen Eintrag im Literaturverzeichnis erzeugt, ohne dass eine 
%                 entsprechende Literaturreferenz im Text der Arbeit angegeben wird
%
%
%


%
% Festlegung des grundlegenden Formatierungsstils des Literaturverzeichnis
%
\bibliographystyle{mystyle}

% Eigentliche Ausgabe der in der Arbeit verwendeten Quellen
%
%
% Angabe der bib-Dateien, in denen die Quellen beschrieben sind;
% die Angabe geht davon aus, dass eine wa.bib-Datei in demselben 
% Verzeichnis liegt, wie se-pa1-vorlage.tex
%
\seBibliography{wa}


%
% Erzeugung der ehrenw\"ortlichen Erkl\"arung
%
% Der optionale Parameter kann verwendet werden, um f\"ur das Thema der Arbeit eine 
% andere Formatierung vorzunehmen; das sollte in der Regel nicht erforderlich sein;
% ausserdem besteht die Gefahr inkonsistenter Titel auf dem Titelblatt und in der 
% ehrenw\"ortlichen Erkl\"arung
%
%\seEhrenwoertlicheErklaerung{} % dieses Kommando sollte standardm\"assig verwendet werden
\seEhrenwoertlicheErklaerung[\LaTeX-Layoutvorlage zur Anfertigung der\\ zweiten Projektarbeit (Version \version)]{}


\end{document}











