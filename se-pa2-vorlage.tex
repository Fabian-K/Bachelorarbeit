%
% Einlesen der .sty-Dateien
%
%  se-pa1-input-styles.tex
%
%  Joerg Baumgart 01.08.2011
%
%  Zusammenfassung und Konfiguration wichtiger Styles f\"ur die 
%  Erzeugung von Seminar-, Projekt- und Bachelorarbeiten
%
%
\documentclass[12pt,BCOR=10mm,headinclude=on,footinclude=off,bibliography=totoc]{scrreprt}
\usepackage[T1]{fontenc}
\usepackage[utf8]{inputenc}
\usepackage[ngerman]{babel} % Deutsche Einstellungen
\usepackage{lmodern}

\usepackage{tikz} % Graphikpaket, das zu pdfLaTeX kompatibel ist
\usepackage{xkeyval} % Definition von Kommandos mit mehreren optionalen Argumenten
\usepackage{listings} % Formatierung von Programmlistings
\usepackage{graphicx} % Einbinden von Graphiken
\usepackage{ifthen}
\usepackage{color}
\usepackage{slashbox} % Diagonalen in Tabellenfeldern
\usepackage{framed} % Erzeugung schwarzer Linien am linken Rand zur Hervorhebung von Textteilen
\usepackage{caption} % Korrektes Setzen einer mehrzeiligen float-Unterschrift bei neu definierten float-Umgebungen
%\usepackage{floatrow}

% Es wird jeweils die sty-Datei importiert und entsprechende Konfigurationseinstellungen werden vorgenommen
%
\usepackage{sty/se-jb-scrpage2} % Formatierung der Kopf- und Fu{\ss}zeilen
\usepackage{sty/se-jb-footmisc}    % Fussnoten besser formatieren

\usepackage{sty/se-jb-glossaries} % Abk\"urzungsverzeichnis, Symbolverzeichnis, Glossar
   
\usepackage{sty/se-jb-floatrow}    % Definition und Konfiguration von float-Umgebungen (figure, table, die neue programm-Umgebung)
% Achtung: se-jb-varioref muss nach se-jb-floatrow importiert werden; 
% andernfalls ist der counter programm f\"ur die labelformat-Anweisung noch nicht definiert   
\usepackage{sty/se-jb-varioref}   % Definition von Querverweisen
\usepackage{sty/se-jb-chngcntr}   % Kapitelweise oder globale Nummerierung von Abbildungen etc.
   
\usepackage{sty/se-jb-listen} % Definition neuer, besser formatierter Listen
\usepackage{sty/se-jb-wa-kommandos} % neue Kommandos f\"ur Seminar-, Projekt- und Bachelorarbeiten


%
% Individuelle Konfiguration des Dokumentes
%
%  Individuelle Konfiguration einer Projektarbeit
%
%
%
%

%
% Literaturverzeichnis
% 
%\usepackage{se-jb-jurabib-theisen} % Literaturverzeichnis gem\"ass den Vorgaben von Theisen aufbauen

\usepackage{bibgerm}
\usepackage{url}
\usepackage{booktabs}
\usepackage{array}

\providecommand{\seCite}[3]{\ifthenelse{\equal{#2}{}}{#1 \cite{#3}}{#1 \cite[#2]{#3}}}


\providecommand{\seFootcite}[3]{\ifthenelse{\equal{#2}{}}{\footnote{#1 \cite{#3}}}{\footnote{#1 \cite[#2]{#3}}}}


% Weitere Optionseinstellungen f\"ur das Koma-Script
%
% Zwischen Abs\"atzen einen Abstand von 0.5 \baselineskip erzeugen
\KOMAoption{parskip}{full}
%
% Vergleiche Duden "Gliederung von Nummern, S.111" 
% DIN 5008 anschauen, wenn sie neu ver\"offentlicht wurde
\KOMAoption{numbers}{noendperiod}
%
%



%  Voreinstellungen f\"ur floats
%  Durch die verwendeten Parameter wird die Wahrscheinlichkeit deutlich kleiner, 
%  dass Gleitobjekte (z. B. Abbildungen) ans Ende des Dokumentes verschoben 
%  werden; 
%  Achtung: clearpage erzwingt die Ausgabe von Gleitobjekten
%
\renewcommand{\topfraction}{1}  % Gleitobjekte d\"urfen eine Seite zu 100% belegen 
\renewcommand{\bottomfraction}{1} % Entsprechender Wert f\"ur den unteren Teil der Seite
\renewcommand{\textfraction}{0} % Eine Seite darf auch ohne Fliesstext existieren
%%%\renewcommand{\floatpagefraction}{1} % Bedeutung unklar, daher keine Ver\"anderung des Vorgabewertes 
                                                                        % von 0.5; eventuell bringt ein \"Anderung auf 1 etwas, wenn 
                                                                         % Probleme mit floats auftreten
                                                                         
                                                                         
                                                                         
% Konfiguration von Programm-Listings
% 
% Achtung: hier gibt es nahezu beliebig viele weitere Konfigurationm\"oglichkeiten; vgl. Paketdokumentation
%
\lstset{language=[R/3 6.10]ABAP,basicstyle=\ttfamily,keywordstyle=\color{blue},captionpos=b,aboveskip=0mm,belowskip=0mm,
          xleftmargin=0em}        
          
%
% Grundkonfiguration der Abs\"ande zwischen den Items der maximal f\"unf Verschachtelungsebenen der 
% neuen Listenumgebungen
%                                                                             
% Initialisierung der Abst\"ande zwischen den items f\"ur seList; Grundeinheit: 0.5\baselineskip; siehe se-jb-listen
\seSetlistbaselineskip{1}{0.75}{0.75}{0.75}{0.75}
% Initialisierung der Abst\"ande zwischen den items f\"ur seToplist; Grundeinheit: 0.5\baselineskip; siehe se-jb-listen
\seSettoplistbaselineskip{1}{0.75}{0.75}{0.75}{0.75}     


%
%  Konfiguration der verschiedenen Verzeichnisse
%
%  abstandEintrag: Wert wird mit \baselineskip multipliziert
%

%
%  Abbildungsverzeichnis
%
\seKonfigurationAbb[
%verzeichnisname=Abbildungsverzeichnis,
labeltextLinks=, % kein Text links;
%labeltextRechts=:,
labelbreite=1cm,
%labeleinzug=1cm,
%abstandEintrag=1,
newpage=ja,
%pnumwidth=20mm,
%dotsep=1000,
%tocrmarg=4.5cm,
%abstandVerzeichnis=-1mm
]

%
% LIstingverzeichnis
%
\seKonfigurationPrg[
%verzeichnisname=Listing-Verzeichnis,
labeltextLinks=,
%labeltextRechts=:,
labelbreite=1cm,
%labeleinzug=2cm,
%abstandEintrag=1,
newpage=ja,
%%pnumwidth=20mm,
%dotsep=1000,
%tocrmarg=4.5cm,
%abstandVerzeichnis=-10mm
]

%
% Tabellenverzeichnis
%
\seKonfigurationTab[
%verzeichnisname=Liste der Tabellen,
labeltextLinks=,
%labeltextRechts=:,
labelbreite=1cm,
%labeleinzug=0.5cm,
%abstandEintrag=1,
newpage=ja,
%pnumwidth=20mm,
%dotsep=1000,
%tocrmarg=4.5cm,
%abstandVerzeichnis=-10mm
]

%
% Abk\"urzungsverzeichnis
%
\seKonfigurationAbk[
%verzeichnisname=Liste der Abk\"urzungen,
%labelbreite=3cm,
%labeleinzug=0.5cm,
%abstandEintrag=1,
%newpage=ja,
%abstandVerzeichnis=-10mm
]

%
% Symbolverzeichnis
% 
\seKonfigurationSym[
%verzeichnisname=Liste der Symbole,
%labelbreite=4cm,
%labeleinzug=3.5cm,
%abstandEintrag=1,
newpage=ja,
%abstandVerzeichnis=-10mm
]

%
% Glossar
%
\seKonfigurationGlo[
%verzeichnisname=Glossar,
%abstandEintrag=0,
]



% (eventuelle) Neudefinition f\"ur die Unter-/\"Uberschriften von Abbildungen, Tabellen und Listings
%
%
%\renewcommand{\seCaptionNameAbbildung}{Abb.}
%\renewcommand{\seCaptionNameTabelle}{Tab.}
%\renewcommand{\seCaptionNameProgramm}{Prg.}


% % (eventuelle) Neudefinition f\"ur Querverweise innerhalb des Textes
%
%
%
%\renewcommand{\seQuerverweisSeite}{Seite}
%\renewcommand{\seQuerverweisAbbildung}{Abb.}
%\renewcommand{\seQuerverweisTabelle}{Tab.}
%\renewcommand{\seQuerverweisProgramm}{Prg.}
%\renewcommand{\seQuerverweisKapitel}{Kap.}
%\renewcommand{\seQuerverweisGleichung}{Gl.}

% Kommandos, die direkt nach \begin{document} ausgef\"uhrt werden m\"ussen
%
%
%
\AtBeginDocument{%
\renewcommand{\listfigurename}{\seAbbildungenVerzeichnisname}
\renewcommand{\listtablename}{\seTabellenVerzeichnisname}
\renewcommand{\figurename}{\seCaptionNameAbbildung}
\renewcommand{\tablename}{\seCaptionNameTabelle}
\pagenumbering{roman}
}
                                                              
                                                                         

%
% Individuelle Definition von Abk\"urzungen, Symbolen und eventuell Glossareintr\"agen
%
%--------------------------------------------------------------------------------------
% Trennungsregeln
%--------------------------------------------------------------------------------------
\hyphenation{prob-lem-los}
\hyphenation{Inter-banken-han-del}
\hyphenation{Business-Objects}
\hyphenation{Web-ser-vice}
\hyphenation{Web-ser-vi-ces}
\hyphenation{Dash-board}
\hyphenation{Dash-boards}
\hyphenation{Zu-gangs-be-schrän-kung-en}
\hyphenation{ge-star-tet}
\hyphenation{NSFR}
\hyphenation{LCR}
\hyphenation{Va-ri-a-blen-aus-prä-gung-en}
\hyphenation{Dia-gramm}
\hyphenation{An-wen-dung}
\hyphenation{An-wen-dung-en}
\hyphenation{An-for-de-rung}
\hyphenation{An-for-de-rung-en}
\hyphenation{Haupt-an-for-de-rung-en}
\hyphenation{tech-nisch-en}
\hyphenation{Funk-ti-o-na-li-tät}
\hyphenation{Or-ches-trie-rung}
\hyphenation{Xcel-si-us}
\hyphenation{Xcel-si-us-A-dap-ter}
\hyphenation{Ein-stel-lung}
\hyphenation{Ein-stel-lung-en}
\hyphenation{Li-qui-di-tät}
\hyphenation{Li-qui-di-täts-ri-si-ko}
\hyphenation{Li-qui-di-täts-ri-si-ko-ma-nage-ment}
\hyphenation{Dash-board}


%--------------------------------------------------------------------------------------
% Abkürzungen
%--------------------------------------------------------------------------------------
\newacronym{dbms}{DBMS}{Da\-ten\-bank\-ma\-nage\-ment\-sys\-tem}
\newacronym{sql}{SQL}{Structured Query Language}
\newacronym{MaRisk}{MaRisk}{Mindestanforderungen an das Risikomanagement}
\newacronym{LiqV}{LiqV}{Liquiditätsverordnung}
\newacronym{ngap}{NGAP}{Next Generation ABAP Plattform}
\newacronym{erp}{ERP}{Enterprise Resource Planning}
\newacronym{crm}{CRM}{Customer Relationship Management}
\newacronym{saplrm}{SAP\,\,LRM}{SAP Liquidity Risk Management}
\newacronym{http}{HTTP}{Hypertext Transfer Protocol}
\newacronym{xml}{XML}{Extensible Markup Language}
\newacronym{NetWeaver}{NetWeaver}{SAP NetWeaver Application Server}
\newacronym{xcelsius}{Xcelsius}{BusinessObjects Xcelsius}
\newacronym{biz}{BIZ}{Bank für Internationalen Zahlungsausgleich}
\newacronym{url}{URL}{Uniform Resource Locator}
\newacronym{wysiwyg}{WYSIWYG}{What You See Is What You Get}
\newacronym{sdk}{SDK}{Software Development Kit}
\newacronym{lcr}{LCR}{Liquidity Coverage Ratio}
\newacronym{nsfr}{NSFR}{Net Stable Funding Ratio}
\newacronym{rest}{REST}{Representational State Transfer}
\newacronym{ria}{RIA}{Rich Internet Application}
\newacronym{abap}{ABAP}{Advanced Business Application Programming}
\newacronym{bw}{BW}{Business Warehouse}
\newacronym{libor}{LIBOR}{London Inter Bank Offered Rate}

%--------------------------------------------------------------------------------------
% Glossareinträge
%--------------------------------------------------------------------------------------

\newglossaryentry{glos:netFramework}{
first=.NET Framework\textsuperscript{GL},
name=.NET Framework,
description={ Das .NET Framework ist eine Plattform von Microsoft, mit der Anwendungen für das Betriebssystem Microsoft Windows erstellt und ausgeführt werden können. Die wichtigsten Komponenten sind Klassenbibliotheken, zum Beispiel für die Entwicklung der Oberflächen, und die Common Language Runtime. Die Anwendungen können in verschiedenen Programmiersprachen geschrieben werden. Zu den unterstützten Sprachen zählen unter anderem C++ und C\#. Der Quellcode wird dann in die Common Intermediate Language compiliert. Diese Zwischensprache kann dann von der Common Language Runtime ausgeführt werden.\seFootcite{vgl.}{S.382ff}{Lou.10}}
}


\newglossaryentry{glos:interbankenhandel}{
first=Inter\-banken\-han\-del\textsuperscript{GL},
name=Interbankenhandel,
description={Der Interbankenhandel ist der Handel von Wertpapieren, Anlagen oder ähnlichem zwischen Banken. Synonym wird auch oft der Begriff Interbankenmarkt verwendet. Für die Zinssätze, mit denen Banken untereinander handeln, existieren anerkannte Referenzen, wie zum Beispiel die \gls{libor}. In Liquiditätsengpässen kann der Interbankenhandel eine wichtige Refinanzierungsrolle darstellen. Der Handel zwischen Banken hängt sehr stark von dem gegenseitigen Vertrauen ab.\seFootcite{vgl.}{S.145f}{Wil.10}}
}

\newglossaryentry{glos:bankrun}{
first=Bankenpanik\textsuperscript{GL},
name=Bankenpanik,
description={Eine Bankenpanik ist ein Ereignis, bei dem eine große Anzahl von Anlegern versucht, ihre Einlagen bei einer Bank abzuziehen. Der Grund kann zum einen in der Veröffentlichung von schlechten Ergebnissen der Bank und damit in einem Vertrauensverlust begründet sein, zum anderen aber auch rein spekulativ sein. Für die Bank besteht die Gefahr der Insolvenz. Im Englischen spricht man von einem Bank Run.\seFootcite{vgl.}{S.1f}{Sch.11}}
}

\newglossaryentry{glos:sqlscript}{
first=SQLScript\textsuperscript{GL},
name=SQLScript,
description={SQLScript ist eine Erweiterung der Abfragesprache SQL und wird in der Datenbank von SAP HANA verwendet. Mit Hilfe von SQLScript lässt sich Anwendungslogik in die Datenbank auslagern. Dazu wurden unter anderem Datentypen, Prozeduren und Kontrollstrukturen hinzugefügt.\seFootcite{vgl.}{S.9f}{SQLScript.11}}
}

\newglossaryentry{glos:ria}{
first=\gls{ria}\textsuperscript{GL},
name=\gls{ria},
sort=RIA,
description={Unter dem Begriff Rich Internet Application werden Webanwendungen bezeichnet, die in ihrer Funktionalität und ihrem Aussehen Desktopanwendungen ähneln. Erstmals eingeführt wurde der Begriff von Macromedia. Zwischen normalen Webanwendungen und RIA kann keine klare Grenze gezogen werden. Ein wichtiges Indiz für eine RIA ist der Einsatz von Technologien wie zum Beispiel Adobe Flash, Adobe Air oder Microsoft Silverlight.\footnote{\seCite{vgl.}{S.32f}{DD.08} \seCite{und}{S.3f}{Pfe.09}\newline
Adobe Flash - http://www.adobe.com/products/flashplayer.html \newline
Adobe Air - http://www.adobe.com/products/air.html \newline
Microsoft Silverlight - http://www.microsoft.com/silverlight/} }
}

\newglossaryentry{glos:bydesign}{
first=SAP Business ByDesign\textsuperscript{GL},
name=SAP Business ByDesign,
description={SAP Business ByDesign ist eine Anwendung von SAP für mittelständige Unternehmen. Zu dem Funktionsunfang gehört sowohl ein \gls{erp}- als auch eine \gls{crm}-Lösung. Die Besonderheit von SAP Business ByDesign ist, dass es auf Servern bei SAP betrieben wird und Kunden die Anwendung mieten und über das Internet konsumieren.}
}

\newglossaryentry{glos:lcr}{
first=\gls{lcr}\textsuperscript{GL},
name=\gls{lcr},
sort=LCR,
description={
Die \gls{lcr}, die auch als Mindestliquiditätsquote bezeichnet wird, ist eine requlatorische Vorgabe im Rahmen von Basel III. Durch die Einhaltung soll sichergestellt werden, dass eine Bank unerwartete Zahlungsmittelabflüsse über einen Zeitraum von 30 Tagen mit eigenen liquiden Aktiva ausgleichen kann.\seFootcite{vgl.}{S.4}{BIS.10}
}
}

\newglossaryentry{glos:nsfr}{
first=\gls{nsfr}\textsuperscript{GL},
name=\gls{nsfr},
sort=NSFR,
description={
Die \gls{nsfr} ist eine requlatorische Vorgabe im Rahmen von Basel III und soll eine mittel- und langfristige Refinanzierung von Banken fördern. Er besteht aus dem Verhältnis von verfügbaren und erforderlichen langfristiger Refinanzierung. Sie wird auch als strukturelle Liquiditätsquote beizeichnet.\seFootcite{vgl.}{S.28}{BIS.10}
}
}

\newglossaryentry{glos:sdk}{
first=\gls{sdk}\textsuperscript{GL},
name=\gls{sdk},
sort=SDK,
description={
Ein \gls{sdk} besteht aus einer Reihe von Anwendungen, mit deren Hilfe Anwendungen erstellt werden können. Meistens gehört zum Umfang eines \gls{sdk} auch eine Dokumentation der Programmierschnittstelle und eine Entwicklungsumgebung dazu.
}
}

\newglossaryentry{glos:atom}{
first=Atom Syndication Format\textsuperscript{GL},
name=Atom Syndication Format,
description={
Das Atom Syndication Format ist ein XML-basierendes Format für den plattformunabhängigen Austausch von Informationen. Es wird in der Regel für sich periodisch änderne Informationen, wie zum Beispiel Newsfeeds, genutzt.\seFootcite{vgl.}{}{ATOM.07}
}
}

\newglossaryentry{glos:tomcat}{
first=Apache Tomcat\textsuperscript{GL},
name=Apache Tomcat,
description={
Der Apache Tomcat ist ein Projekt der Apache Software Foundation, mit dem Java-Code auf Webservern ausgeführt werden kann. Dadurch können dynamische Webanwendungen entwickelt werden.
}
}

\newglossaryentry{glos:rest}{
first=\gls{rest}\textsuperscript{GL},
name=\gls{rest},
sort=REST,
description={
\gls{rest} ist ein Paradigma für die Umsetzung von Webservices. Dabei sieht es vor, dass eine Ressource eindeutig über einen Uniform Resource Identifier identifiziert werden kann. Diese Ressource kann in verschiedenen Repräsentationen abgerufen werden. Auf Ressourcen lassen sich festgelegte Operationen anwenden. Die Kommunikation findet immer zustandslos statt.\seFootcite{vgl.}{S.245f}{RRD.07}
}
}

\newglossaryentry{glos:actionscript}{
first=ActionScript\textsuperscript{GL},
name=ActionScript,
description={
ActionScript ist die Programmiersprache für die Entwicklung von Adobe Flex-basierenden Programmen. ActionScript unterstützt die objektorientierte Programmierung und basiert auf dem Sprachkern von JavaScript. Für die Entwicklung existiert mit dem Flex Builder eine integrierte Entwicklungsumgebung.\seFootcite{vgl.}{S.3f}{Moo.07}
}
}


\newglossaryentry{glo:abap}{
first=\gls{abap}\textsuperscript{GL},
name=\gls{abap},
sort=ABAP,
description={
\gls{abap} ist eine Programmiersprache der vierten Generation. Sie wird für einen Großteil der von SAP entwickelten Anwendungen verwendet. Eine Besonderheit von \gls{abap} ist die einfache Integration von \gls{sql}-Abfragen. Mit \gls{abap} Objects ist auch die objektorientierte Programmierung möglich.
}
} 

\usepackage{bibgerm}

\begin{document}

\newcommand{\version}{0.1}

% Erzeugung des Titelblatts
%
%
%
\seTitelblattZweiteProjektarbeit[
%hilfslinien=ja,
%dhbwlogoSkalierung=0.5,
%dhbwlogoDeltaX=2.4,
%dhbwlogoDeltaY=-10,
firmenlogo=images/sap-logo,
%firmenlogoSkalierung=0.5,
firmenlogoDeltaX=-13,
firmenlogoDeltaY=-10,
thema=Entwicklung einer Zwischenschicht für die Nutzung weiterer Anwendungen in Verbindung mit der Berechnungskomponente des Liquidity Risk Managements,
verfasser=Fabian Kajzar,
%verfasserin=,
matrikelnummer=428094,
kurs=WWI\,09\,SW\,B,
firma=SAP AG,
abteilung=Application Strategic Innovation - High Performance Applications,
%studiengangsleiterin=,
studiengangsleiter=Prof. Dr.-Ing. J\"org Baumgart,
%wissenschaftlicheBetreuerinName=Prof. Dr. Hans-Henning Pagnia,
%wissenschaftlicheBetreuerinEmail=hans-henning.pagnia@dhbw-mannheim.de,
%wissenschaftlicheBetreuerinTelefon=0621 4105-1131,
wissenschaftlicherBetreuerName=Prof. Dr. Hans-Henning Pagnia,
wissenschaftlicherBetreuerEmail=hans-henning.pagnia@dhbw-mannheim.de,
wissenschaftlicherBetreuerTelefon=0621 4105-1131,
%firmenbetreuerinName=
%firmenbetreuerinEmail=
%firmenbetreuerinTelefon=
firmenbetreuerName=Jens Mett,
firmenbetreuerEmail=jens.mett@sap.com,
firmenbetreuerTelefon=06227 7-61785,
bearbeitungszeitraumVon=13. Februar 2012,
bearbeitungszeitraumBis=4. Mai 2012,
sperrvermerk=nein,
%sperrvermerkAlternativerText={Ein (inhaltlich) alternativer Text darf nur verwendet werden, wenn er vom Studiengangsleiter genehmigt wurde!}
]



% Erzeugung der Kurzfassung; Verfasser, Firma und Thema werden automatisch \"ubernommen
%
% Der optionale Parameter kann verwendet werden, um f\"ur das Thema der Arbeit eine 
% andere Formatierung vorzunehmen; das sollte in der Regel nicht erforderlich sein;
% ausserdem besteht die Gefahr inkonsistenter Titel auf dem Titelblatt und in der 
% Kurzfassung
%
\seKurzfassung{} % dieses Kommando sollte standardm\"assig verwendet werden
%\seKurzfassung[\LaTeX-Layoutvorlage zur Anfertigung der zweiten Projektarbeit (Version \version)]{}

% Beispiel f\"ur ein Kapitel, dass vor dem Einleitungskapitel kommt, z. B. ein Vorwort oder eine Danksagung
\seKapitelVorEinleitung{Vorwort}

Diese \LaTeX{}-Vorlage soll es erm\"oglichen, ohne tiefergehende \LaTeX-Kenntnisse 
eine \seArbeit{} erstellen zu k\"onnen, die die Empfehlungen und Vorgaben aus dem 
Dokument \textbf{Empfehlungen und Hinweise zur Anfertigung der zweiten Projektarbeit (Version 0.92)}
erf\"ullt. 

Das vorliegende Dokument besitzt die Version \version, da die zugeh\"origen sty-Dateien 
nach und nach erg\"anzt werden, um f\"ur Seminar-, Projekt- und Bachelorarbeiten neue 
Kommandos f\"ur eine effiziente Dokumenterstellung anbieten zu k\"onnen. 
Neue Versionen werden abw\"artskompatibel sein. 



% Ausgabe des Inhaltsverzeichnisses
%
%
\seInhaltsverzeichnis[%
einrueckung=ja,
gliederungsebenen=4
]



% Ausgabe der verschiedenen Verzeichnisse
% abk: Abk\"urzungsverzeichnis
% sym: Symbolverzeichnis
% abb: Abbildungsverzeichnis
% tab: Tabellenverzeichnis
% prg: Listingverzeichnis
%
%
% Achtung: Abk\"urzungs- und Symbolverzeichnis werden nur ausgegeben, wenn mindest ein Symbol bzw. 
%                mindestens eine Abk\"urzung in der Arbeit verwendet wurden
%
%
% gliederungsebene:
% -- section: die Verzeichnisse werden einem Kapitel "Verzeichnisse" untergliedert
% -- chapter: die Verzeichnisse sind jeweils eigene Kapitel
% imInhaltsverzeichnis: ja/nein -- Sollen die Verzeichnisse im Inhaltsverzeichnis enthalten sein?
\seVerzeichnisse[gliederungsebene=section,imInhaltsverzeichnis=ja]{abk}{sym}{abb}{tab}{prg}






% Erstes eigentliches Kapitel der Arbeit; typischerweise das Einleitungskapitel;
% hier muss wieder auf die Nummierung mit arabischen Seitenzahlen umgestellt werden
%
\chapter{Einleitung}\pagenumbering{arabic}


test \seFootcite{vgl.}{S.5}{test}

% Erstes Hauptkapitel der Arbeit 
%
%
%
\chapter{Der formale Aufbau einer Projektarbeit}
% Mit markright kann eine verk\"urzte Version der \"Uberschrift f\"ur den Seitenkopf generiert werden
%
%
%\markright{Formaler Aufbau}




% Anhang der Arbeit
% 
%
\seAppendix{}
\chapter{Einige wichtige \LaTeX{}-Kommandos}

minor change

%  Testdatei f\"ur die Erzeugung von Literaturreferenzen, die den Regeln von Rene Theisen 
%  (Wissenschaftliches Arbeiten, 2009) folgen
%
%
%
\section{Kommandos f\"ur die Erzeugung von Literaturverweisen}

Test

%
% Ein kleiner Text, um Abk\"urzungen, Symbole und Glossareintr\"age zu testen
%
%
\section{Kommandos f\"ur die Erzeugung von Abk\"urzungen, Symbolen und Glos\-sar\-eint\-r\"a\-gen}

asd

\section{Abbildungen, Tabellen und Programmlistings\label{gleitobjekte}}

asd

\section{Die Definition und Anwendung von zwei neuen Listenumgebungen}
asd


\chapter{Hinweise zur Installation und \"Ubersetzung}

\section{Verwendung von TeXShop (Apple-Welt)}

Unter den ausgelieferten Dateien befinden sich zwei \textbf{engine}-Dateien: 

\begin{seList}
\item \verb+dhbw-projektarbeit.engine+
\item \verb+dhbw-projektarbeit-remove-all.engine+ (l\"oscht alle erzeugten \textsl{Hilfsdateien})
\end{seList}

Mit jeder dieser beiden Dateien kann man die Vorlage \verb+se-pa2-vorlage.tex+ 
\"ubersetzen. Alle Verzeichnisse (insbesondere Abk\"urzungs- und Symbolverzeichnis) 
sowie das Glossar werden (hoffentlich) korrekt erstellt.  

In den engine-Dateien ist beschrieben, an welcher Stelle sie im Mac OS X Dateisystem 
installiert werden m\"ussen, damit man sie direkt von TeXShop aus aufrufen kann. 

\section{Verwendung von MiKTeX (Windows-Welt)}

F\"ur die \"Ubersetzung wird eine batch-Datei \verb+make-projektarbeit.bat+ zur Verf\"ugung 
gestellt, mit der man in der Windows-\textsl{Eingabeaufforderung} (cmd) die Vorlage \"ubersetzen kann. Der Aufruf lautet:
\verb+make-projektarbeit.bat se-pa2-vorlage+

Da MiKTeX eine andere Version von \verb+jurabib+ verwendet, mit der sich die 
Vorlage nicht korrekt \"ubersetzen l\"asst, werden die beiden Dateien 

\begin{seList}
\item \verb+jurabib.sty+ und 
\item \verb+jurabib.bst+
\end{seList}

aus der TeX Live Version von Mac OS X mitgeliefert. Damit sollte die 
\"Ubersetzung problemlos funktionieren. 


 

%Dann hoffe ich mal, dass sich mit den Vorlagen etwas anfangen l�sst. Sie sind (absichtlich) in 
%
%einer Version 0.9, da ich an den zugeh�rigen sty-Dateien weitere Erg�nzungen vornehmen werde,
%
%um f�r zuk�nftige Arbeiten neue komfortable Kommandos zur Verf�gung zu stellen. 

%
%  Erzeugung eines Glossars
%
% Achtung: Das Glossar wird nur ausgegeben, wenn mindestens ein Eintrag in der Arbeit 
%                definiert wurde
%
%
\newpage
\sePrintGlossary{}


%
% Literaturverzeichnisses
%
%\newpage
\sePrintBibliography{}

%  Erzeugung von Eintr\"agen im Literaturverzeichnis
%
%  Achtung: in einer Projektarbeit darf da \nocite-Kommando nicht verwendet werden,
%                 da es einen Eintrag im Literaturverzeichnis erzeugt, ohne dass eine 
%                 entsprechende Literaturreferenz im Text der Arbeit angegeben wird
%
%
%


%
% Festlegung des grundlegenden Formatierungsstils des Literaturverzeichnis
%
\bibliographystyle{mystyle}

% Eigentliche Ausgabe der in der Arbeit verwendeten Quellen
%
%
% Angabe der bib-Dateien, in denen die Quellen beschrieben sind;
% die Angabe geht davon aus, dass eine wa.bib-Datei in demselben 
% Verzeichnis liegt, wie se-pa1-vorlage.tex
%
\seBibliography{wa}


%
% Erzeugung der ehrenw\"ortlichen Erkl\"arung
%
% Der optionale Parameter kann verwendet werden, um f\"ur das Thema der Arbeit eine 
% andere Formatierung vorzunehmen; das sollte in der Regel nicht erforderlich sein;
% ausserdem besteht die Gefahr inkonsistenter Titel auf dem Titelblatt und in der 
% ehrenw\"ortlichen Erkl\"arung
%
%\seEhrenwoertlicheErklaerung{} % dieses Kommando sollte standardm\"assig verwendet werden
\seEhrenwoertlicheErklaerung[\LaTeX-Layoutvorlage zur Anfertigung der\\ zweiten Projektarbeit (Version \version)]{}


\end{document}











