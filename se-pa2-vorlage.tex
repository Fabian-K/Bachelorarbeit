% -------------------------------------------------------------------------------------
% Einlesen der .sty-Dateien
% -------------------------------------------------------------------------------------
%  se-pa1-input-styles.tex
%
%  Joerg Baumgart 01.08.2011
%
%  Zusammenfassung und Konfiguration wichtiger Styles f\"ur die 
%  Erzeugung von Seminar-, Projekt- und Bachelorarbeiten
%
%
\documentclass[12pt,BCOR=10mm,headinclude=on,footinclude=off,bibliography=totoc]{scrreprt}
\usepackage[T1]{fontenc}
\usepackage[utf8]{inputenc}
\usepackage[ngerman]{babel} % Deutsche Einstellungen
\usepackage{lmodern}

\usepackage{tikz} % Graphikpaket, das zu pdfLaTeX kompatibel ist
\usepackage{xkeyval} % Definition von Kommandos mit mehreren optionalen Argumenten
\usepackage{listings} % Formatierung von Programmlistings
\usepackage{graphicx} % Einbinden von Graphiken
\usepackage{ifthen}
\usepackage{color}
\usepackage{slashbox} % Diagonalen in Tabellenfeldern
\usepackage{framed} % Erzeugung schwarzer Linien am linken Rand zur Hervorhebung von Textteilen
\usepackage{caption} % Korrektes Setzen einer mehrzeiligen float-Unterschrift bei neu definierten float-Umgebungen
%\usepackage{floatrow}

% Es wird jeweils die sty-Datei importiert und entsprechende Konfigurationseinstellungen werden vorgenommen
%
\usepackage{sty/se-jb-scrpage2} % Formatierung der Kopf- und Fu{\ss}zeilen
\usepackage{sty/se-jb-footmisc}    % Fussnoten besser formatieren

\usepackage{sty/se-jb-glossaries} % Abk\"urzungsverzeichnis, Symbolverzeichnis, Glossar
   
\usepackage{sty/se-jb-floatrow}    % Definition und Konfiguration von float-Umgebungen (figure, table, die neue programm-Umgebung)
% Achtung: se-jb-varioref muss nach se-jb-floatrow importiert werden; 
% andernfalls ist der counter programm f\"ur die labelformat-Anweisung noch nicht definiert   
\usepackage{sty/se-jb-varioref}   % Definition von Querverweisen
\usepackage{sty/se-jb-chngcntr}   % Kapitelweise oder globale Nummerierung von Abbildungen etc.
   
\usepackage{sty/se-jb-listen} % Definition neuer, besser formatierter Listen
\usepackage{sty/se-jb-wa-kommandos} % neue Kommandos f\"ur Seminar-, Projekt- und Bachelorarbeiten


% -------------------------------------------------------------------------------------
% Individuelle Konfiguration des Dokumentes
% -------------------------------------------------------------------------------------
%  Individuelle Konfiguration einer Projektarbeit
%
%
%
%

%
% Literaturverzeichnis
% 
%\usepackage{se-jb-jurabib-theisen} % Literaturverzeichnis gem\"ass den Vorgaben von Theisen aufbauen

\usepackage{bibgerm}
\usepackage{url}
\usepackage{booktabs}
\usepackage{array}

\providecommand{\seCite}[3]{\ifthenelse{\equal{#2}{}}{#1 \cite{#3}}{#1 \cite[#2]{#3}}}


\providecommand{\seFootcite}[3]{\ifthenelse{\equal{#2}{}}{\footnote{#1 \cite{#3}}}{\footnote{#1 \cite[#2]{#3}}}}


% Weitere Optionseinstellungen f\"ur das Koma-Script
%
% Zwischen Abs\"atzen einen Abstand von 0.5 \baselineskip erzeugen
\KOMAoption{parskip}{full}
%
% Vergleiche Duden "Gliederung von Nummern, S.111" 
% DIN 5008 anschauen, wenn sie neu ver\"offentlicht wurde
\KOMAoption{numbers}{noendperiod}
%
%



%  Voreinstellungen f\"ur floats
%  Durch die verwendeten Parameter wird die Wahrscheinlichkeit deutlich kleiner, 
%  dass Gleitobjekte (z. B. Abbildungen) ans Ende des Dokumentes verschoben 
%  werden; 
%  Achtung: clearpage erzwingt die Ausgabe von Gleitobjekten
%
\renewcommand{\topfraction}{1}  % Gleitobjekte d\"urfen eine Seite zu 100% belegen 
\renewcommand{\bottomfraction}{1} % Entsprechender Wert f\"ur den unteren Teil der Seite
\renewcommand{\textfraction}{0} % Eine Seite darf auch ohne Fliesstext existieren
%%%\renewcommand{\floatpagefraction}{1} % Bedeutung unklar, daher keine Ver\"anderung des Vorgabewertes 
                                                                        % von 0.5; eventuell bringt ein \"Anderung auf 1 etwas, wenn 
                                                                         % Probleme mit floats auftreten
                                                                         
                                                                         
                                                                         
% Konfiguration von Programm-Listings
% 
% Achtung: hier gibt es nahezu beliebig viele weitere Konfigurationm\"oglichkeiten; vgl. Paketdokumentation
%
\lstset{language=[R/3 6.10]ABAP,basicstyle=\ttfamily,keywordstyle=\color{blue},captionpos=b,aboveskip=0mm,belowskip=0mm,
          xleftmargin=0em}        
          
%
% Grundkonfiguration der Abs\"ande zwischen den Items der maximal f\"unf Verschachtelungsebenen der 
% neuen Listenumgebungen
%                                                                             
% Initialisierung der Abst\"ande zwischen den items f\"ur seList; Grundeinheit: 0.5\baselineskip; siehe se-jb-listen
\seSetlistbaselineskip{1}{0.75}{0.75}{0.75}{0.75}
% Initialisierung der Abst\"ande zwischen den items f\"ur seToplist; Grundeinheit: 0.5\baselineskip; siehe se-jb-listen
\seSettoplistbaselineskip{1}{0.75}{0.75}{0.75}{0.75}     


%
%  Konfiguration der verschiedenen Verzeichnisse
%
%  abstandEintrag: Wert wird mit \baselineskip multipliziert
%

%
%  Abbildungsverzeichnis
%
\seKonfigurationAbb[
%verzeichnisname=Abbildungsverzeichnis,
labeltextLinks=, % kein Text links;
%labeltextRechts=:,
labelbreite=1cm,
%labeleinzug=1cm,
%abstandEintrag=1,
newpage=ja,
%pnumwidth=20mm,
%dotsep=1000,
%tocrmarg=4.5cm,
%abstandVerzeichnis=-1mm
]

%
% LIstingverzeichnis
%
\seKonfigurationPrg[
%verzeichnisname=Listing-Verzeichnis,
labeltextLinks=,
%labeltextRechts=:,
labelbreite=1cm,
%labeleinzug=2cm,
%abstandEintrag=1,
newpage=ja,
%%pnumwidth=20mm,
%dotsep=1000,
%tocrmarg=4.5cm,
%abstandVerzeichnis=-10mm
]

%
% Tabellenverzeichnis
%
\seKonfigurationTab[
%verzeichnisname=Liste der Tabellen,
labeltextLinks=,
%labeltextRechts=:,
labelbreite=1cm,
%labeleinzug=0.5cm,
%abstandEintrag=1,
newpage=ja,
%pnumwidth=20mm,
%dotsep=1000,
%tocrmarg=4.5cm,
%abstandVerzeichnis=-10mm
]

%
% Abk\"urzungsverzeichnis
%
\seKonfigurationAbk[
%verzeichnisname=Liste der Abk\"urzungen,
%labelbreite=3cm,
%labeleinzug=0.5cm,
%abstandEintrag=1,
%newpage=ja,
%abstandVerzeichnis=-10mm
]

%
% Symbolverzeichnis
% 
\seKonfigurationSym[
%verzeichnisname=Liste der Symbole,
%labelbreite=4cm,
%labeleinzug=3.5cm,
%abstandEintrag=1,
newpage=ja,
%abstandVerzeichnis=-10mm
]

%
% Glossar
%
\seKonfigurationGlo[
%verzeichnisname=Glossar,
%abstandEintrag=0,
]



% (eventuelle) Neudefinition f\"ur die Unter-/\"Uberschriften von Abbildungen, Tabellen und Listings
%
%
%\renewcommand{\seCaptionNameAbbildung}{Abb.}
%\renewcommand{\seCaptionNameTabelle}{Tab.}
%\renewcommand{\seCaptionNameProgramm}{Prg.}


% % (eventuelle) Neudefinition f\"ur Querverweise innerhalb des Textes
%
%
%
%\renewcommand{\seQuerverweisSeite}{Seite}
%\renewcommand{\seQuerverweisAbbildung}{Abb.}
%\renewcommand{\seQuerverweisTabelle}{Tab.}
%\renewcommand{\seQuerverweisProgramm}{Prg.}
%\renewcommand{\seQuerverweisKapitel}{Kap.}
%\renewcommand{\seQuerverweisGleichung}{Gl.}

% Kommandos, die direkt nach \begin{document} ausgef\"uhrt werden m\"ussen
%
%
%
\AtBeginDocument{%
\renewcommand{\listfigurename}{\seAbbildungenVerzeichnisname}
\renewcommand{\listtablename}{\seTabellenVerzeichnisname}
\renewcommand{\figurename}{\seCaptionNameAbbildung}
\renewcommand{\tablename}{\seCaptionNameTabelle}
\pagenumbering{roman}
}
                                                              
                                                                         

% -------------------------------------------------------------------------------------
% Individuelle Definition von Abk\"urzungen, Symbolen und eventuell Glossareintr\"agen
% -------------------------------------------------------------------------------------
%--------------------------------------------------------------------------------------
% Trennungsregeln
%--------------------------------------------------------------------------------------
\hyphenation{prob-lem-los}
\hyphenation{Inter-banken-han-del}
\hyphenation{Business-Objects}
\hyphenation{Web-ser-vice}
\hyphenation{Web-ser-vi-ces}
\hyphenation{Dash-board}
\hyphenation{Dash-boards}
\hyphenation{Zu-gangs-be-schrän-kung-en}
\hyphenation{ge-star-tet}
\hyphenation{NSFR}
\hyphenation{LCR}
\hyphenation{Va-ri-a-blen-aus-prä-gung-en}
\hyphenation{Dia-gramm}
\hyphenation{An-wen-dung}
\hyphenation{An-wen-dung-en}
\hyphenation{An-for-de-rung}
\hyphenation{An-for-de-rung-en}
\hyphenation{Haupt-an-for-de-rung-en}
\hyphenation{tech-nisch-en}
\hyphenation{Funk-ti-o-na-li-tät}
\hyphenation{Or-ches-trie-rung}
\hyphenation{Xcel-si-us}
\hyphenation{Xcel-si-us-A-dap-ter}
\hyphenation{Ein-stel-lung}
\hyphenation{Ein-stel-lung-en}
\hyphenation{Li-qui-di-tät}
\hyphenation{Li-qui-di-täts-ri-si-ko}
\hyphenation{Li-qui-di-täts-ri-si-ko-ma-nage-ment}
\hyphenation{Dash-board}


%--------------------------------------------------------------------------------------
% Abkürzungen
%--------------------------------------------------------------------------------------
\newacronym{dbms}{DBMS}{Da\-ten\-bank\-ma\-nage\-ment\-sys\-tem}
\newacronym{sql}{SQL}{Structured Query Language}
\newacronym{MaRisk}{MaRisk}{Mindestanforderungen an das Risikomanagement}
\newacronym{LiqV}{LiqV}{Liquiditätsverordnung}
\newacronym{ngap}{NGAP}{Next Generation ABAP Plattform}
\newacronym{erp}{ERP}{Enterprise Resource Planning}
\newacronym{crm}{CRM}{Customer Relationship Management}
\newacronym{saplrm}{SAP\,\,LRM}{SAP Liquidity Risk Management}
\newacronym{http}{HTTP}{Hypertext Transfer Protocol}
\newacronym{xml}{XML}{Extensible Markup Language}
\newacronym{NetWeaver}{NetWeaver}{SAP NetWeaver Application Server}
\newacronym{xcelsius}{Xcelsius}{BusinessObjects Xcelsius}
\newacronym{biz}{BIZ}{Bank für Internationalen Zahlungsausgleich}
\newacronym{url}{URL}{Uniform Resource Locator}
\newacronym{wysiwyg}{WYSIWYG}{What You See Is What You Get}
\newacronym{sdk}{SDK}{Software Development Kit}
\newacronym{lcr}{LCR}{Liquidity Coverage Ratio}
\newacronym{nsfr}{NSFR}{Net Stable Funding Ratio}
\newacronym{rest}{REST}{Representational State Transfer}
\newacronym{ria}{RIA}{Rich Internet Application}
\newacronym{abap}{ABAP}{Advanced Business Application Programming}
\newacronym{bw}{BW}{Business Warehouse}
\newacronym{libor}{LIBOR}{London Inter Bank Offered Rate}

%--------------------------------------------------------------------------------------
% Glossareinträge
%--------------------------------------------------------------------------------------

\newglossaryentry{glos:netFramework}{
first=.NET Framework\textsuperscript{GL},
name=.NET Framework,
description={ Das .NET Framework ist eine Plattform von Microsoft, mit der Anwendungen für das Betriebssystem Microsoft Windows erstellt und ausgeführt werden können. Die wichtigsten Komponenten sind Klassenbibliotheken, zum Beispiel für die Entwicklung der Oberflächen, und die Common Language Runtime. Die Anwendungen können in verschiedenen Programmiersprachen geschrieben werden. Zu den unterstützten Sprachen zählen unter anderem C++ und C\#. Der Quellcode wird dann in die Common Intermediate Language compiliert. Diese Zwischensprache kann dann von der Common Language Runtime ausgeführt werden.\seFootcite{vgl.}{S.382ff}{Lou.10}}
}


\newglossaryentry{glos:interbankenhandel}{
first=Inter\-banken\-han\-del\textsuperscript{GL},
name=Interbankenhandel,
description={Der Interbankenhandel ist der Handel von Wertpapieren, Anlagen oder ähnlichem zwischen Banken. Synonym wird auch oft der Begriff Interbankenmarkt verwendet. Für die Zinssätze, mit denen Banken untereinander handeln, existieren anerkannte Referenzen, wie zum Beispiel die \gls{libor}. In Liquiditätsengpässen kann der Interbankenhandel eine wichtige Refinanzierungsrolle darstellen. Der Handel zwischen Banken hängt sehr stark von dem gegenseitigen Vertrauen ab.\seFootcite{vgl.}{S.145f}{Wil.10}}
}

\newglossaryentry{glos:bankrun}{
first=Bankenpanik\textsuperscript{GL},
name=Bankenpanik,
description={Eine Bankenpanik ist ein Ereignis, bei dem eine große Anzahl von Anlegern versucht, ihre Einlagen bei einer Bank abzuziehen. Der Grund kann zum einen in der Veröffentlichung von schlechten Ergebnissen der Bank und damit in einem Vertrauensverlust begründet sein, zum anderen aber auch rein spekulativ sein. Für die Bank besteht die Gefahr der Insolvenz. Im Englischen spricht man von einem Bank Run.\seFootcite{vgl.}{S.1f}{Sch.11}}
}

\newglossaryentry{glos:sqlscript}{
first=SQLScript\textsuperscript{GL},
name=SQLScript,
description={SQLScript ist eine Erweiterung der Abfragesprache SQL und wird in der Datenbank von SAP HANA verwendet. Mit Hilfe von SQLScript lässt sich Anwendungslogik in die Datenbank auslagern. Dazu wurden unter anderem Datentypen, Prozeduren und Kontrollstrukturen hinzugefügt.\seFootcite{vgl.}{S.9f}{SQLScript.11}}
}

\newglossaryentry{glos:ria}{
first=\gls{ria}\textsuperscript{GL},
name=\gls{ria},
sort=RIA,
description={Unter dem Begriff Rich Internet Application werden Webanwendungen bezeichnet, die in ihrer Funktionalität und ihrem Aussehen Desktopanwendungen ähneln. Erstmals eingeführt wurde der Begriff von Macromedia. Zwischen normalen Webanwendungen und RIA kann keine klare Grenze gezogen werden. Ein wichtiges Indiz für eine RIA ist der Einsatz von Technologien wie zum Beispiel Adobe Flash, Adobe Air oder Microsoft Silverlight.\footnote{\seCite{vgl.}{S.32f}{DD.08} \seCite{und}{S.3f}{Pfe.09}\newline
Adobe Flash - http://www.adobe.com/products/flashplayer.html \newline
Adobe Air - http://www.adobe.com/products/air.html \newline
Microsoft Silverlight - http://www.microsoft.com/silverlight/} }
}

\newglossaryentry{glos:bydesign}{
first=SAP Business ByDesign\textsuperscript{GL},
name=SAP Business ByDesign,
description={SAP Business ByDesign ist eine Anwendung von SAP für mittelständige Unternehmen. Zu dem Funktionsunfang gehört sowohl ein \gls{erp}- als auch eine \gls{crm}-Lösung. Die Besonderheit von SAP Business ByDesign ist, dass es auf Servern bei SAP betrieben wird und Kunden die Anwendung mieten und über das Internet konsumieren.}
}

\newglossaryentry{glos:lcr}{
first=\gls{lcr}\textsuperscript{GL},
name=\gls{lcr},
sort=LCR,
description={
Die \gls{lcr}, die auch als Mindestliquiditätsquote bezeichnet wird, ist eine requlatorische Vorgabe im Rahmen von Basel III. Durch die Einhaltung soll sichergestellt werden, dass eine Bank unerwartete Zahlungsmittelabflüsse über einen Zeitraum von 30 Tagen mit eigenen liquiden Aktiva ausgleichen kann.\seFootcite{vgl.}{S.4}{BIS.10}
}
}

\newglossaryentry{glos:nsfr}{
first=\gls{nsfr}\textsuperscript{GL},
name=\gls{nsfr},
sort=NSFR,
description={
Die \gls{nsfr} ist eine requlatorische Vorgabe im Rahmen von Basel III und soll eine mittel- und langfristige Refinanzierung von Banken fördern. Er besteht aus dem Verhältnis von verfügbaren und erforderlichen langfristiger Refinanzierung. Sie wird auch als strukturelle Liquiditätsquote beizeichnet.\seFootcite{vgl.}{S.28}{BIS.10}
}
}

\newglossaryentry{glos:sdk}{
first=\gls{sdk}\textsuperscript{GL},
name=\gls{sdk},
sort=SDK,
description={
Ein \gls{sdk} besteht aus einer Reihe von Anwendungen, mit deren Hilfe Anwendungen erstellt werden können. Meistens gehört zum Umfang eines \gls{sdk} auch eine Dokumentation der Programmierschnittstelle und eine Entwicklungsumgebung dazu.
}
}

\newglossaryentry{glos:atom}{
first=Atom Syndication Format\textsuperscript{GL},
name=Atom Syndication Format,
description={
Das Atom Syndication Format ist ein XML-basierendes Format für den plattformunabhängigen Austausch von Informationen. Es wird in der Regel für sich periodisch änderne Informationen, wie zum Beispiel Newsfeeds, genutzt.\seFootcite{vgl.}{}{ATOM.07}
}
}

\newglossaryentry{glos:tomcat}{
first=Apache Tomcat\textsuperscript{GL},
name=Apache Tomcat,
description={
Der Apache Tomcat ist ein Projekt der Apache Software Foundation, mit dem Java-Code auf Webservern ausgeführt werden kann. Dadurch können dynamische Webanwendungen entwickelt werden.
}
}

\newglossaryentry{glos:rest}{
first=\gls{rest}\textsuperscript{GL},
name=\gls{rest},
sort=REST,
description={
\gls{rest} ist ein Paradigma für die Umsetzung von Webservices. Dabei sieht es vor, dass eine Ressource eindeutig über einen Uniform Resource Identifier identifiziert werden kann. Diese Ressource kann in verschiedenen Repräsentationen abgerufen werden. Auf Ressourcen lassen sich festgelegte Operationen anwenden. Die Kommunikation findet immer zustandslos statt.\seFootcite{vgl.}{S.245f}{RRD.07}
}
}

\newglossaryentry{glos:actionscript}{
first=ActionScript\textsuperscript{GL},
name=ActionScript,
description={
ActionScript ist die Programmiersprache für die Entwicklung von Adobe Flex-basierenden Programmen. ActionScript unterstützt die objektorientierte Programmierung und basiert auf dem Sprachkern von JavaScript. Für die Entwicklung existiert mit dem Flex Builder eine integrierte Entwicklungsumgebung.\seFootcite{vgl.}{S.3f}{Moo.07}
}
}


\newglossaryentry{glo:abap}{
first=\gls{abap}\textsuperscript{GL},
name=\gls{abap},
sort=ABAP,
description={
\gls{abap} ist eine Programmiersprache der vierten Generation. Sie wird für einen Großteil der von SAP entwickelten Anwendungen verwendet. Eine Besonderheit von \gls{abap} ist die einfache Integration von \gls{sql}-Abfragen. Mit \gls{abap} Objects ist auch die objektorientierte Programmierung möglich.
}
} 

\begin{document}

\newcommand{\version}{0.1}

% -------------------------------------------------------------------------------------
% Erzeugung des Titelblatts
% -------------------------------------------------------------------------------------
\seTitelblattZweiteProjektarbeit[
firmenlogo=images/sap-logo,
firmenlogoDeltaX=-13,
firmenlogoDeltaY=-10,
thema=Entwicklung einer Zwischenschicht für die Nutzung weiterer Anwendungen in Verbindung mit der Berechnungskomponente des Liquidity Risk Managements,
verfasser=Fabian Kajzar,
matrikelnummer=428094,
kurs=WWI\,09\,SW\,B,
firma=SAP AG,
abteilung=Application Strategic Innovation - HPA,
studiengangsleiter=Prof. Dr.-Ing. J\"org Baumgart,
wissenschaftlicherBetreuerName=Prof. Dr. Hans-Henning Pagnia,
wissenschaftlicherBetreuerEmail=hans-henning.pagnia@dhbw-mannheim.de,
wissenschaftlicherBetreuerTelefon=0621 4105-1131,
firmenbetreuerName=Jens Mett,
firmenbetreuerEmail=jens.mett@sap.com,
firmenbetreuerTelefon=06227 7-61785,
bearbeitungszeitraumVon=13. Februar 2012,
bearbeitungszeitraumBis=4. Mai 2012,
sperrvermerk=nein
]

\pagenumbering{Roman}
\setcounter{page}{0}

% -------------------------------------------------------------------------------------
% Erzeugung der Kurzfassung; Verfasser, Firma und Thema werden automatisch \"ubernommen
% -------------------------------------------------------------------------------------
\seKurzfassung{}

\glsresetall

% Beispiel f\"ur ein Kapitel, dass vor dem Einleitungskapitel kommt, z. B. ein Vorwort oder eine Danksagung
%\seKapitelVorEinleitung{Vorwort}

% -------------------------------------------------------------------------------------
% Ausgabe des Inhaltsverzeichnisses
% -------------------------------------------------------------------------------------
\seInhaltsverzeichnis[
einrueckung=ja,
gliederungsebenen=4
]

% -------------------------------------------------------------------------------------
% Ausgabe der verschiedenen Verzeichnisse
% -------------------------------------------------------------------------------------
\seVerzeichnisse[gliederungsebene=section,imInhaltsverzeichnis=ja]{abk}{sym}{abb}{tab}{prg}


% -------------------------------------------------------------------------------------
% Vorbereitung für Inhalt der Arbeit
% -------------------------------------------------------------------------------------


% -------------------------------------------------------------------------------------
% Inhalt der Arbeit
% -------------------------------------------------------------------------------------
\chapter{Einleitung}
\pagenumbering{arabic}

\chapter{Grundlagen}

\section{Liquidität}

Der Begriff der Liquidität ist weit verbreitet und im allgemeinen Sprachgebrauch festgesetzt. Allerdings ist eine eindeutige Definition des Begriffs schwierig, da Liquidität sehr vielschichtig ist, mehrere Dimensionen besitzt und die jeweilige Bedeutung von der Perspektive der Betrachtung abhängt.\footnote{\seCite{vgl.}{S.3}{Dur.11} \seCite{und}{S.13}{Bar.08}} Für diese Arbeit ist vor allem die betriebswirtschaftliche Sicht auf Liquidität entscheidend, die volkswirtschaftliche Sicht wird daher nicht näher erläutert.\seFootcite{vgl.}{S.10}{ADF.10}

In der betriebswirtschaftlichen Sicht kann wird zunächst die Liquidität von Objekten von der Liquidität von Subjekten unterschieden. Die Objektliquidität ist die Eigenschaft eines Vermögensgegenstandes in Zahlungsmittel umwandeln zu können.\seFootcite{vgl.}{S.10}{Moc.07} Sie hängt demnach von der Nähe des Objektes zu Geld ab, Zahlungsmittel haben die höchste Objektliquidität, Immobilien eine geringe.\seFootcite{vgl.}{S.3}{Dur.11} Die Liquidität von Subjekten bezeichnet die Fähigkeit eines Subjekts, z.B. einer Bank, alle Zahlungsverpflichtungen erfüllen zu können \footnote{\seCite{vgl.}{S.3}{Dur.11} \seCite{und}{S.11}{ADF.10}}

Zeitlich kann Liquidität in kurz und langfristig unterschieden werden. Bei der kurzfristigen Liquidität steht der Zahlungsaspekt im Vordergrund, meist nur auf einen Tag bezogen.\seFootcite{vgl.}{S.3f}{Dur.11}. Es muss zu jeder Zeit sichergestellt werden, dass alle fälligen Zahlungen in der entsprechenden Höhe beglichen werden können. Diese Bedingung ist bei der Steuerung von Banken zu jedem Zeitpunkt streng einzuhalten. \footnote{\seCite{vgl.}{S.13}{Bar.08} \seCite{und}{S.12}{ADF.10}}. Synonym werden auch die Begriffe operative Liquidität sowie dispositive Liquidität verwendet.\seFootcite{vgl.}{S.13}{Bar.08}

Die langfristige Liquidität bezeichnet die Fähigkeit langfristige Refinanzierungsmittel auf der Passiv-Seite der Bilanz aufzunehmen um dadurch die gewünschte Entwicklung auf der Aktiv-Seite der Bilanz ermöglichen zu können. Sie ist also mit den Zielen des Subjektes verknüpft.\seFootcite{vgl.}{S.4}{Dur.11} Für Banken ist dies besonders wichtig, da es einen wichtigen Wettbewerbsvorteil gegenüber Konkurrenzen darstellt\seFootcite{vgl.}{S.13}{Bar.08} Zwischen der kurz und langfristigen Liquidität besteht eine beidseitige Wechselwirkung, eine niedrige kurzfristige Liquidität führt zu Problemen bei der langfristigen Liquidität.\seFootcite{vgl.}{S.15}{Bar.08}

Die Folgen der Liquidität können weitreichend sein. Probleme mit sowohl der kurzfristigen als auch der langfristigen Liquidität können zu einem Reputationsverlust führen. Gerade bei Banken hat dies schwere Auswirkungen, da Fremdkapitalgeber das Vertrauen in die Bank verlieren. Dies wiederum hat Auswirkungen auf die Passiv-Seite der Bilanz, viel Fremdkapital wird verloren gehen. Im schlimmsten Fall, wenn die Bank ihren Zahlungsverpflichtung nicht mehr nachkommen kann, muss sie Insolvenz anmelden.\footnote{\seCite{vgl.}{S.4}{Dur.11} \seCite{und}{S.65}{Rom.10}}

\section{Liquiditätsrisiko}
Die Finanzinstitute haben in der Vergangenheit dem Liquiditätsrisiko keine besondere Bedeutung zugewandt. Ob ein Institut das Risiko gesondert behandelt hat oder nicht konnte frei gewählt werden. Erst im Jahr 2007, als die Grundstückspreise in den USA zusammengebrochen sind und dadurch viele Banken in Liquiditätsschwierigkeiten gekommen sind, rückte die Behandlung des Liquiditätsrisikos in den Fokus - nicht zuletzt durch die Pleite der Lehman Brothers Bank  [TODO ref?]\seFootcite{vgl.}{S.5}{Bar.08}

Das Liquiditätsrisiko ist das Risiko, gegenwärtige oder zukünftige Zahlungsverpflichtungen entweder nicht, nicht vollständig oder nicht zeitgerecht nachkommen zu können.\footnote{\seCite{vgl.}{S.467f}{Hul.10} \seCite{, }{S.167}{Rom.10} \seCite{und}{S.6}{Dur.11}}. Grundsätzlich ist das Liquiditätsrisiko bei allen Unternehmen vorhanden. Bei Banken ist es allerdings besonders stark ausgeprägt, da hier sowohl die Ein- als auch die Auszahlungen in hohem Maße von dem Kundenverhalten abhängen.\footnote{\seCite{vgl.}{S.90}{ADF.10} \seCite{und}{S.79}{Bar.08}}. Im weitesten Sinne wird zu dem Liquiditätsrisiko auch die Opportunitätskosten hinzugezogen, die entstehen wenn eine gewinnbringende Transaktion aufgrund fehlender Zahlungsmittel nicht durchgeführt werden kann.\seFootcite{vgl.}{S.79}{Bar.08}

Analog zu der Unterteilung des Liquiditätsbegriffes kann auch das Liquiditätsrisiko weiter unterteilt werden. Zunächst unterscheidet man in dem bankenbezogenen Liquiditätsrisiko das Liquiditätsspannungsrisiko und das Zahlungsmittelbedarfsrisiko.

Das Liquiditätsanpassungsrisiko beinhaltet grundsätzlich Risiken aufgrund von Zuflüssen und kann in das Refinanzierungsrisiko und das Marktliquiditätsrisiko unterteilt werden.\seFootcite{vgl.}{S.7}{Dur.11}. Wenn im Falle eines Engpass nicht genügend Mittel beschafft werden können, oder dies nur unter erhöhten Marktpreisen erreicht werden kann wird von dem Refinanzierungsrisiko gesprochen. Das Vertrauen der Marktteilnehmer ist hier entscheidend, beeinflusst werden kann es vor allem durch die Veränderung des Leitzinses der Notenbank.\seFootcite{vgl.}{S.7f}{Dur.11} Das Marktliquiditätsrisiko bezieht sich auf die Geldnähe von Aktiva (siehe TODO->Liquidität) und bezeichnet das Risiko, einen Aktivposten nur zu hohen Transaktionskosten liquidieren zu können. Es ist nur schwer beeinflussbar, da es von dem aktuellen Angebot und der Nachfrage auf dem jeweiligen Markt abhängt. \seFootcite{vgl.}{S.9}{Dur.11}

Das Zahlungsmittelbedarfsrisiko, auch originäres Liquiditätsrisiko genannt, beruht im Gegensatz zu dem Liquiditätsspannungsrisiko auf den Abflüssen von Liquidität. Es wird hauptsächlich das Terminrisiko und das Abrufrisiko unterschieden.\footnote{\seCite{vgl.}{S.7f}{Dur.11} \seCite{und}{S.12}{ADF.10}} Das Terminrisiko resultiert aus verspäteten Zahlungseingängen, genauer gesagt aus außerplanmäßigen Prolongationen von Aktivgeschäften über die vereinbarte Kapitalbindungsdauer hinaus.\footnote{\seCite{vgl.}{S.12}{Poh.08} \seCite{und}{S.51}{Zer.05}}. Ein Beispiel ist die Verlängerung eines Kredites, da der Kreditnehmer die Tilgung oder die Zinsen des Kredites nicht bezahlen kann.\seFootcite{vgl.}{S.10}{Dur.11} Das Abrufrisiko beruht auf einer unerwarteten Ausnutzung von zugesagten Kreditlinien. Hier findet ein Liquiditätsabfluss in unerwarteter Höhe statt.\seFootcite{vgl.}{S.513f}{SLK.08} Der bekannteste und zugleich extremste Fall des Abrufrisikos ist eine \gls{glos:bankrun}.

\section{Liquiditätsrisikomanagement}

\chapter{SAP LRM und Xcelsius}
\section{Einleitung}
\section{SAP LRM}
\subsection{Funktionen}
\subsection{Architektur}
\subsubsection{NGAP}
\newpage
\subsubsection{HANA}
Die \gls{HANA} ist ein Produkt der SAP und besteht aus Softwarekomponenten, die in Kombination mit zertifizierter Hardware verkauft werden. Es ist die Reaktion auf den Bedarf nach der schnellen Auswertung von großen Datenmengen. Dies soll durch die Ausnutzung der Leistungssteigerung von modernen Computern erreicht werden. Hier ist zum einen die Entwicklung von Einkernprozessoren zu Mehrkernprozessoren zu nennen und zum Anderen die Verfügbarkeit von schnellem Hauptspeicher in der benötigten Größe zu vertretbaren Kosten.\seFootcite{vgl.}{S.14f}{PZ.11} Das Ziel von \gls{HANA} ist es aktuelle operationale Daten in Verbindung mit bestehenden historischen Daten in Echtzeit zu Analysieren und somit Informationen zu gewinnen.\seFootcite{vgl.}{}{HANA.12}

Der Kern der \gls{HANA} bildet dabei ein hauptspeicherbasiertes \gls{DBMS}. Dabei werden alle Daten nicht wie bei traditionellen \gls{DBMS}en auf Festplatten gespeichert, sondern im Hauptspeicher gehalten um höhere Zugriffsgeschwindigkeiten zu erreichen.\seFootcite{vgl.}{TODO}{Kle.10} Außerdem ist neben der zeilenbasierten Organisation der Daten im Speicher auch die spaltenbasierte Organisation möglich. Die zeilenbasierte Organisation ist von Vorteil, wenn auf einzelne Datensätze komplett zugegriffen werden soll, die spaltenbasierte Organisation ist bei Tabellen mit einer hohen Anzahl an Spalten und bei spaltenbasierten Operationen wie der Aggregation oder der Suche überlegen. Durch die Unterstützung von beiden Organisationsformen kann die jeweils beste Form gewählt werden.\seFootcite{vgl.}{S.13f}{Kle.10}

Veränderungen in einem Datensatz einer Tabelle können auf Wunsch nicht in dem Eintrag der Tabelle direkt geändert, sondern nur die Differenzen an die Tabelle angefügt werden. Dadurch bleibt die Information, wie sich der Datensatz im Laufe der Zeit verändert hat, erhalten und kann in späteren Auswertungen als weitere Information hinzugezogen werden. Zusätzlich ist das Anfügen der Veränderung schneller durchzuführen wie die Veränderung des bestehenden Datensatzes.\seFootcite{vgl.}{S.109f}{PZ.11}

Zu den genannten Veränderungen wird in Anwendungen, die auf Basis von \gls{HANA} entwickelt werden, versucht, ein Teil der Anwendungslogik schon auf der Datenbank selbst zu berechnen.\seFootcite{vgl.}{S.155f}{PZ.11} Erreicht wird dies durch die Erweiterung der Abfragesprache \gls{SQL} zu \gls{glos:sqlscript}. Mit \gls{glos:sqlscript} ist es unter Anderem durch das Hinzufügen von Datentypen, Prozeduren und Operationen möglich, Anwendungslogik abzubilden. Diese Berechnungen können von der Datenbank durch Parallelisierung sehr schnell durchgeführt werden.\seFootcite{vgl.}{S.9f}{SQLScript.11} Als Resultat kann die Datenübertragung zwischen dem \gls{DBMS} und der Anwendung verringert werden, da nur noch das Ergebnis und nicht die Datensätze, auf denen das Ergebnis basiert, übertragen werden muss und die Komplexität der Anwendung verringert werden, da ein Teil der Logik von dem \gls{DBMS} übernommen wird.


\subsubsection{Oberon}
\subsection{Berechnungskomponente}
\section{Xcelsius}
\subsection{Grundprinzip}
\subsection{Funktionen}
\subsection{Architektur}
\subsection{Erweiterungsmöglichkeiten}
\section{Zusammenfassung}

\chapter{Anforderung}
\section{Einleitung}
\section{Ziel}
\section{Anwendungsfälle}
\section{Zusammenfassung}

\chapter{Umsetzungsmöglichkeiten}
\section{Einleitung}
\section{WebService}
\section{Zusammenfassung}

\chapter{Umsetzung}
\section{Einleitung}
\section{Analyse}
\section{Entwurf}
\section{Implementierung}
\section{Zusammenfassung}

\chapter{Evaluation}
\section{Einleitung}
\section{Möglichkeiten}
\section{Vergleich}
\section{Performance}
\section{Zusammenfassung}

\chapter{Zusammenfassung}


% -------------------------------------------------------------------------------------
% Anhang der Arbeit
% -------------------------------------------------------------------------------------
\seAppendix{}

\pagenumbering{Roman}
\setcounter{page}{8}

\chapter{Anhang}

Inhalt des Anhangs

% -------------------------------------------------------------------------------------
%  Erzeugung eines Glossars
% -------------------------------------------------------------------------------------
\newpage
\sePrintGlossary{}

% -------------------------------------------------------------------------------------
% Literaturverzeichnisses
% -------------------------------------------------------------------------------------
\sePrintBibliography{}

\bibliographystyle{sty/bibliothek-style}
\seBibliography{literatur}

% -------------------------------------------------------------------------------------
% Erzeugung der ehrenw\"ortlichen Erkl\"arung
% -------------------------------------------------------------------------------------
\seEhrenwoertlicheErklaerung{}

\end{document}
