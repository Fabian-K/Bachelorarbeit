\section{Verwendung von TeXShop (Apple-Welt)}

Unter den ausgelieferten Dateien befinden sich zwei \textbf{engine}-Dateien: 

\begin{seList}
\item \verb+dhbw-projektarbeit.engine+
\item \verb+dhbw-projektarbeit-remove-all.engine+ (l\"oscht alle erzeugten \textsl{Hilfsdateien})
\end{seList}

Mit jeder dieser beiden Dateien kann man die Vorlage \verb+se-pa2-vorlage.tex+ 
\"ubersetzen. Alle Verzeichnisse (insbesondere Abk\"urzungs- und Symbolverzeichnis) 
sowie das Glossar werden (hoffentlich) korrekt erstellt.  

In den engine-Dateien ist beschrieben, an welcher Stelle sie im Mac OS X Dateisystem 
installiert werden m\"ussen, damit man sie direkt von TeXShop aus aufrufen kann. 

\section{Verwendung von MiKTeX (Windows-Welt)}

F\"ur die \"Ubersetzung wird eine batch-Datei \verb+make-projektarbeit.bat+ zur Verf\"ugung 
gestellt, mit der man in der Windows-\textsl{Eingabeaufforderung} (cmd) die Vorlage \"ubersetzen kann. Der Aufruf lautet:
\verb+make-projektarbeit.bat se-pa2-vorlage+

Da MiKTeX eine andere Version von \verb+jurabib+ verwendet, mit der sich die 
Vorlage nicht korrekt \"ubersetzen l\"asst, werden die beiden Dateien 

\begin{seList}
\item \verb+jurabib.sty+ und 
\item \verb+jurabib.bst+
\end{seList}

aus der TeX Live Version von Mac OS X mitgeliefert. Damit sollte die 
\"Ubersetzung problemlos funktionieren. 


 

%Dann hoffe ich mal, dass sich mit den Vorlagen etwas anfangen l�sst. Sie sind (absichtlich) in 
%
%einer Version 0.9, da ich an den zugeh�rigen sty-Dateien weitere Erg�nzungen vornehmen werde,
%
%um f�r zuk�nftige Arbeiten neue komfortable Kommandos zur Verf�gung zu stellen. 